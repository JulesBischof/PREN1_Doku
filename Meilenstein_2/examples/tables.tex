% --------------- Basic Table # Simple table with horizontal and vertical lines.

\begin{table}[h]                                    % h = here
    \centering
    \begin{tabular}{|c|c|c|}                        % c=centered, l=left, r=right
        \hline
        Column 1 & Column 2 & Column 3 \\ \hline
        Row 1    & Data 1   & Data 2   \\ \hline
        Row 2    & Data 3   & Data 4   \\ \hline
    \end{tabular}
    \caption{A basic table}
    \label{tab:basic}
\end{table}



% --------------- Table with Multirow for spanning rows # Table with Multirow: Includes cells that span multiple rows using the multirow package.

\begin{table}[h]
    \centering
    \begin{tabular}{|c|c|c|}
        \hline
        \multirow{2}{*}{Rowspan} & Column 1 & Column 2 \\ \cline{2-3}
                                 & Data 1   & Data 2   \\ \hline
        Row 2                    & Data 3   & Data 4   \\ \hline
    \end{tabular}
    \caption{Table with multirow cells}
    \label{tab:multirow}
\end{table}

\multirow{2}{*}{Rowspan}        % creates a cell that spans two rows.
\cline{2-3}                     % creates a horizontal line spanning columns 2 and 3.



% ------------- Table with Fixed Column Widths # Table with Fixed Column Widths: Uses p{width} for fixed-width columns and text wrapping.

\begin{table}[h]
    \centering
    \begin{tabular}{|p{3cm}|p{5cm}|}
        \hline
        Short Column & Long Column                                                                 \\ \hline
        Text         & This is a longer text that will wrap to fit within the width of the column. \\ \hline
        More Text    & Another example of a long text that will wrap.                              \\ \hline
    \end{tabular}
    \caption{Table with fixed column widths}
    \label{tab:fixedwidth}
\end{table}

p{3cm} % specifies a column width of 3 cm and allows text to wrap within that width.



% ------------- Table with Multi-column # Table with Multi-column: Merges multiple columns into one using \multicolumn.

\begin{table}[h]
    \centering
    \begin{tabular}{|c|c|c|}
        \hline
        \multicolumn{2}{|c|}{Merged Columns} & Column 3          \\ \hline
        Row 1                                & Data 1   & Data 2 \\ \hline
        Row 2                                & Data 3   & Data 4 \\ \hline
    \end{tabular}
    \caption{Table with merged columns}
    \label{tab:multicolumn}
\end{table}

\multicolumn{2}{|c|}{Merged Columns} % merges two columns into one with centered text.



% ------------- Table with Booktabs # Complex Table with Booktabs: Uses the booktabs package for a more polished look.

\begin{table}[h]
    \centering
    \begin{tabular}{lcc}
        \toprule
        Column 1 & Column 2 & Column 3 \\ \midrule
        Row 1    & Data 1   & Data 2   \\
        Row 2    & Data 3   & Data 4   \\
        Row 3    & Data 5   & Data 6   \\ \bottomrule
    \end{tabular}
    \caption{Table with booktabs}
    \label{tab:booktabs}
\end{table}

\toprule, \midrule, \bottomrule % create top, middle, and bottom horizontal lines, respectively.



% ------------- get Colour in Tables

\usepackage[table]{xcolor} % Load xcolor with the table option



% -------------  Creating a Basic Table with Colored Rows

\begin{table}[h]
    \centering
    \begin{tabular}{|c|c|c|}
        \hline
        \rowcolor{gray!30} % Light gray background color for the header row
        Column 1 & Column 2 & Column 3 \\ \hline
        \rowcolor{yellow!20} % Light yellow background color for this row
        Row 1 & Data 1 & Data 2 \\ \hline
        \rowcolor{cyan!10} % Light cyan background color for this row
        Row 2 & Data 3 & Data 4 \\ \hline
        \rowcolor{white} % Default color (no background color)
        Row 3 & Data 5 & Data 6 \\ \hline
    \end{tabular}
    \caption{Table with colored rows}
    \label{tab:coloredrows}
\end{table}



% -------------  Creating a Basic Table with Colored Columns

\begin{table}[h]
    \centering
    \begin{tabular}{|>{\columncolor{red!10}}c|>{\columncolor{green!10}}c|>{\columncolor{blue!10}}c|}
        \hline
        Column 1 & Column 2 & Column 3 \\ \hline
        Row 1 & Data 1 & Data 2 \\ \hline
        Row 2 & Data 3 & Data 4 \\ \hline
        Row 3 & Data 5 & Data 6 \\ \hline
    \end{tabular}
    \caption{Table with colored columns}
    \label{tab:coloredcolumns}
\end{table}

% ------------- automatisch zählende Zeilennummerierung

\newcounter{rownum} % define Rowcounter

\begin{tabularx}{\textwidth}{|>{\centering\arraybackslash}p{0.5cm}|>{\raggedright\arraybackslash}X|>{\raggedright\arraybackslash}X|>{\raggedright\arraybackslash}X|>{\raggedright\arraybackslash}X|}
    \hline
    \textbf{\#} & \textbf{Risiko} & \textbf{SA} & \textbf{EW} & \textbf{Auswirkungen} \\
    \hline
    \stepcounter{rownum}\arabic{rownum} & \lipsum[1][1-5] & \lipsum[2][1-3] & \lipsum[3][1-3] & \lipsum[4][1-4] \\
    \hline
    \stepcounter{rownum}\arabic{rownum} & \lipsum[1][1-6] & \lipsum[2][1-4] & \lipsum[3][1-2] & \lipsum[4][1-5] \\
    \hline
    \stepcounter{rownum}\arabic{rownum} & \lipsum[1][1-3] & \lipsum[2][1-6] & \lipsum[3][1-4] & \lipsum[4][1-2] \\
    \hline
\end{tabularx}


gray!30: 30%        -- gray color.
yellow!20: 20%      -- yellow color.
cyan!10: 10%        -- cyan color.
white: Default color (no background).