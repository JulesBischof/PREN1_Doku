\newcolumntype{L}[1]{>{\raggedright\arraybackslash}p{#1}}  % Left-aligned column with fixed width
\newcolumntype{C}[1]{>{\centering\arraybackslash}p{#1}}    % Centered column with fixed width
\newcolumntype{R}[1]{>{\raggedleft\arraybackslash}p{#1}}   % Right-aligned column with fixed width

\setlength{\parindent}{0pt} % kein Einzug nach Absatz
% \setlength{\parindent}{15pt} % Standardmässiger Einzug
% \noindent % Temporär keinen Einzug

\usetikzlibrary{matrix, backgrounds} 

\newcommand\zeilenabstand{10pt} 

\def\GetNodeColumn(#1-#2-#3){#3}

\tikzset{vp/.style={circle, fill, inner sep=2pt}} 
\newcommand\verbindungslinie[4][10pt]{ 
  \node[#2, vp] at ([yshift=-.15*\zeilenabstand]#3.south) {};
  \edef\co{\noexpand\GetNodeColumn(#3)}
  \foreach[remember=\p as \lastp (initially #3)] \p in {#4} 
    \edef\cl{\noexpand\GetNodeColumn(\lastp)}
    \edef\cp{\noexpand\GetNodeColumn(\p)}
    \draw[ultra thick, #2, draw opacity=.4]
        ([yshift=-.15*\zeilenabstand,xshift={#1*(\co-\cl)}]\lastp.south) -- 
        ([yshift=-.15*\zeilenabstand,xshift={#1*(\co-\cp)}]\p.south) 
        node[vp] {}; 
} 