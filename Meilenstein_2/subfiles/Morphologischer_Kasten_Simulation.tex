% \documentclass[DIV=21,12pt,captions=tableheading,landscape,a4paper]{scrreprt}
% \usepackage{booktabs}
% \usepackage{tikz} 




\documentclass[main.tex]{subfiles} % Subfile-Class

\begin{document}

\begin{landscape} % Querformat beginnen
  \thispagestyle{fancy}

  % ======================================================= Moprhologischer Kasten Simulation
  \subsection{Simulation}

  \subsubsection{Morphologischer Kasten Simulation}

  \begin{tikzpicture}
    \matrix[
      matrix of nodes,
      inner sep=0pt,
      row sep=\zeilenabstand,
      column sep=15pt,
      nodes={font=\strut},
      nodes in empty cells,
      column 2/.style={text width=3.75cm},
      column 3/.style={text width=3cm,align=center},
      column 4/.style={text width=3cm,align=center},
      column 5/.style={text width=3cm,align=center},
      column 6/.style={text width=3cm,align=center},
      column 7/.style={text width=3cm,align=center}
    ](m){
       & Betriebssystem
       & Windows
       & Linux
       & MacOS
       & Alle OS
       & -                              \\
       & Darstellung
       & 2-D
       & 3-D
       & mixed Reality
       & 2-D mit Sensorfeedback
       & -                              \\
       & Pfadfindungsalgo
       & Mehrere Algos
       & A*-Algorithmus
       & D*Lite
       & Partikelfilter
       & -                              \\
       & Fahrzeugparameter
       & editierbar
       & fix
       & lernfähige
       & dynamisch
       & -                              \\
       & Daten-Export
       & Log-File
       & CSV
       & SQ-Lite
       & JSON
       & -                              \\
       & Programmiersprache Frontend
       & JavaScript
       & Java (JavaFX)
       & C++(QT)
       & Python(Tkinter)
       & -                              \\
       & Programmiersprache Backend
       & JavaScript
       & Java
       & C++
       & Python
       & -                              \\
       & UI Technologie
       & Webapplikation
       & Desktop App
       & CLI
       & 3-D Simulation mit Interaktion
       & -                              \\
       & Auswertung
       & Diagramme in MS Excel
       & Log-Auswertung im Tool
       & Live Dashboard mit Updates
       & -
       & -                              \\
       & Simulations-geschwindigkeit
       & Echtzeit
       & Zeitraffer
       & Verlangsamte Bewegung
       & Anpassbar
       & -                              \\
       & Interaktions-möglichkeiten
       & Passiv beobachten
       & Eingreifen per Klick
       & Eingreifen via Konsole
       & Vollinteraktives Umfeld
       & -                              \\
       & Fahrverhalten
       & konstant
       & dynamisch (anpassbar)
       & lernfähig
       & -
       & -                              \\
    };
    % Kopfzeile 
    \node(ul)[anchor=south west]
    at ([yshift={\zeilenabstand+\aboverulesep+\belowrulesep}]m.north west)
    {Parameter};
    \node(or)[anchor=south east] at (ul.north-|m-1-2.east){Ideen};
    \foreach[count=\i from 3] \l in {1,2,3,4,5}
    \node[anchor=base] at (or.base-|m-1-\i){\l};
    % Tabellenlinien 
    \draw[line width=\lightrulewidth](or.north-|ul.west)--(or.east|-ul.south)
    ([yshift=-\aboverulesep]ul.south-|m.west)
    --([yshift=-\aboverulesep]ul.south-|m.east);
    \draw[line width=\heavyrulewidth]([yshift=\belowrulesep]or.north-|m.west)
    --([yshift=\belowrulesep]or.north-|m.east)
    ([yshift={-\aboverulesep}]m.south west)
    --([yshift={-\aboverulesep}]m.south east);
    % Verbindungslinien 
    \begin{scope}[on background layer]
      \verbindungslinie{green}{m-1-4}{m-2-3,m-3-5,m-4-4,m-5-6,m-6-4,m-7-4,m-8-4,m-9-5,m-10-3,m-11-3,m-12-5}
      \verbindungslinie{red}{m-1-6}{m-2-3,m-3-3,m-4-3,m-5-5,m-6-3,m-7-3,m-8-3,m-9-5,m-10-6,m-11-3,m-12-4}
    \end{scope}
  \end{tikzpicture}
  \captionof{table}{Morphologischer Kasten Simulation}\label{tab:morphKasten_simulation}

\end{landscape} % Querformat beenden

\end{document}