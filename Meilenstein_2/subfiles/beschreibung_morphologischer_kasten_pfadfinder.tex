\documentclass[main.tex]{subfiles} % Subfile-Class


% ============================================================================== %
%                            Subfile document                                    %
% ============================================================================== %

\begin{document}

% Template

\subsubsection{Beschreibung der Lösungsvarianten Pfadfinder}
Im folgenden Abschnitt werden die drei Varianten aus dem morphologischen Kasten
schriftlich festgehalten.

\subsubsection*{Rote Variante}
Die rote Variante bahnt sich den Weg durch das Netzwerk, indem diese immer zu
den nächsten Nodes vorausschaut. Es ist daher ein probabilistischer
Algorithmus, welcher das Netzwerk bei der Fahrt aufzeichnet. Die verschiedenen
Nodes sowie die Verbindungslinien können mit einem Liniensensor aus einem
Phototransistor-Array detektiert werden. Damit das Fahrzeug auf der Linie
bleibt und nicht den Fliesenfugen nachfährt, wird ein zweites
Phototransistor-Array als Liniensensor eingebaut. Bevor eine Strecke befahren
wird, wird bereits mit einem LIDAR geprüft, ob eine Pylone auf dem nächsten
Wegpunkt sein wird. Es wird ebenfalls der Unterschied zwischen einer Pylone und
einem Hindernis mit dem LIDAR erkannt. Dafür wird der Höhenunterschied beider
Objekte ausgenutzt. Um Kollisionen mit Hindernissen auf der Strecke zu
verhindern und diese gleichzeitig gezielt zu lokalisieren, wird eine
Lichtschranke verwendet, welche durch das Hindernis unterbrochen wird. Damit
die aktuelle Position im Graph möglichst exakt bestimmt werden kann, wird ein
Gyroskop für die Winkelbestimmung verwendet, sowie die gefahrene Strecke
gemessen. Mit beiden Werten kann der Graph grob nachkonstruiert werden. Somit
werden der Graph und die fehlenden Linien probabilistisch bestimmt. Die
Antriebe sowie die Sensoren werden auf einem Microcontroller angesteuert. Die
Steuersignale werden mit einem High-Level-Controller(\textit{HLC}), z.B. einem
Raspberry Pi, berechnet.

Der Energiespeicher ist ein Lithium-Akku und der Akkustand wird durch eine
LCD-Anzeige angezeigt. Wenn das Ziel erreicht wird, wird dies mit einem
Piezo-Alarm ausgegeben. Für die Zielerkennung wird eine Kamera eingesetzt,
welche die Beschriftung der Nodes erkennen kann.

Das Fahrzeug fährt mit zwei Gummirädern und wird mit
Schrittmotoren angetrieben. Einen dritten Auflagepunkt bildet eine nachlaufende Kugel. Die Hindernis-Aufnahmevorrichtung
wird mit einem Klemmgreifer, welcher von oben ohne Winkelausrichtung zupackt,
realisiert. Bei der Umplatzierung des Hindernisses dreht sich das Fahrzeug um
180 Grad und positioniert es am korrekten Ort anhand der gefahrenen Distanz und
des Liniensensors. Nach dem Absetzen des Hindernisses fährt es noch ein Stück
rückwärts und dreht sich wieder um 180 Grad zurück. Die Z-Positionierung wird
mit einem Linearmodul realisiert. Das Fahrzeug kann im Notfall durch einen Buzzer
gestoppt werden.

\subsubsection*{Blaue Variante}
Die Pfadfindung in der blauen Variante ist ähnlich der roten Variante. Jedoch
werden in der blauen Variante die Nodes mittels einer Kamera detektiert. Die
Kamera wird ausserdem verwendet, um den Weg durch das Netzwerk zu bestimmen.
Hierzu wird am Start ein Bild des Netzwerks erstellt. Mithilfe dieses Bildes
wird der Weg entsprechend den gesperrten Wegpunkten, Hindernissen und der
Strecke gewichtet und möglichst optimal abgefahren. Der Rest der
Umwelterfassung ist identisch mit der roten Variante.

Der Energiespeicher ist ein Lithium-Akku, welcher den Energiestand durch eine
LCD-Anzeige angibt. Die Zielerkennung wird durch eine reine Navigation
ermöglicht und die Zielankunft wird mit dem Piezo-Alarm mitgeteilt.

Das Fahrzeug wird mit einer mittleren Achse mit zwei Gummirädern gelenkt und
hat vorne und hinten je eine Kugel zur Stützung. Für den Antrieb des Fahrzeugs
wird ein DC-Motor verwendet. Ein Hindernis wird mit einem Gabelstapler durch
die jeweiligen Löcher im Hindernis aufgenommen. Das Fahrzeug hat eine
symmetrische Bauform und kann deshalb nach dem Absetzen des Hindernisses
Rückwärts weiterfahren. Durch die gefahrene Distanz und durch den Liniensensor
weiss das Fahrzeug exakt, wo es das Hindernis absetzen muss. Die
Z-Positionierung wird mit einem Linearmodul ermöglicht. Im Notfall kann das
Fahrzeug mit einem Buzzer gestoppt werden.

\subsubsection*{Grüne Variante}
Bei der grünen Variante wird auf Brute-Force gesetzt, also durch ein rein
systematisches Ausprobieren aller möglichen Lösungen soll ein Weg durch das
Netzwerk gefunden werden. Die Nodes, Linien und die Fliesenfugen werden mittels
Bilderkennung erfasst und unterschieden. Auf ein Detektieren der fehlenden
Linien wird verzichtet. Auch die Hindernisse und Pylonen werden mit
Bilderkennung erkannt und unterschieden. Die aktuelle Position im Netzwerk wird
durch den Fahrwinkel und die gefahrene Strecke laufend berechnet. Um
Kollisionen mit einem Hindernis zu verhindern, wird eine Lichtschranke
verwendet. Aufgrund der Brute-Force-Taktik wird bei der Kartenerkennung nur auf
auftretende Ereignisse reagiert. Weil eine hohe Rechenleistung anfällt, wird
die gesamte Regelung auf einem HLC realisiert.

Ein Nickel-Cadmium-Akku versorgt das Fahrzeug mit genug Energie. Der Akku-Stand
wird mit mehreren LEDs angezeigt. Die Ankunft im Ziel wird durch die
Bilderkennung erkannt und mit einem Lautsprecher ausgegeben.

Damit das Fahrzeug möglichst manövrierfähig ist, werden Mecanumräder mit einer
Panzerlenkung verwendet. Als Antrieb werden DC-Motoren genommen. Die
Hindernis-Aufnahmevorrichtung ist ein Klemmgreifer, welcher rotierbar ist.
Somit wird das Hindernis um das Fahrzeug herumgedreht. Das Hindernis wird mit
der Kamera positioniert. Die Z-Positionierung wird mit einer Hebebühne
realisiert. Das Fahrzeug kann jederzeit mit einem Buzzer gestoppt werden.

\end{document}
