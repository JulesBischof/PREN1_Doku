\documentclass[main.tex]{subfiles} % Subfile-Class


% ============================================================================== %
%                            Subfile document                                    %
% ============================================================================== %

\begin{document}

% Template

\subsubsection{Beschreibung der Lösungsvarianten Simulation}
Im folgenden Abschnitt werden die zwei Varianten aus dem morphologischen Kasten für die Simulation
schriftlich festgehalten.

\subsubsection*{Rote Variante}
Die rote Variante der Simulation ist so konzipiert, dass sie auf allen gängigen Betriebssystemen (Windows, Linux und MacOS) lauffähig ist. 
Dadurch wird eine plattformunabhängige Entwicklung und Testung gewährleistet. Die Darstellung erfolgt in 2D, was eine klare und einfache 
Visualisierung der Simulation ermöglicht. Ein besonderes Merkmal dieser Variante ist die Möglichkeit, mehrere Pfadfindungsalgorithmen zu 
testen. Dies erlaubt es, den besten Algorithmus für die jeweilige Problemstellung zu evaluieren und die Effizienz des Fahrzeugs zu optimieren. 
Die Fahrzeugparameter sind editierbar, sodass Anpassungen vorgenommen werden können, um verschiedene Szenarien zu simulieren. Die Daten der 
Simulation werden in einer SQLite-Datenbank gespeichert, was eine dauerhafte und strukturierte Speicherung ermöglicht. Von dort aus können 
die Daten einfach exportiert und weiterverarbeitet werden. Für die Benutzeroberfläche wird ein JavaScript-Framework eingesetzt (Svelte), 
um eine moderne und reaktionsschnelle Webanwendung zu bieten. Im Gegensatz zu anderen Varianten wird in der roten Variante kein Unterschied 
zwischen Frontend und Backend gemacht; die gesamte Simulation läuft innerhalb eines Webinterfaces mit Svelte. Zur Analyse der Simulationsergebnisse 
steht ein Dashboard zur Verfügung, über das verschiedene Metriken und Auswertungen eingesehen werden können. Die Simulationsgeschwindigkeit 
ist anpassbar, was es ermöglicht, die Simulation in unterschiedlichen Geschwindigkeiten zu testen und dadurch das Verhalten unter verschiedenen 
Bedingungen zu analysieren. Während der Simulation ist kein direkter Eingriff möglich, sodass die Simulation lediglich passiv beobachtet werden 
kann. Das Fahrverhalten des Fahrzeugs ist jedoch dynamisch anpassbar, was bedeutet, dass verschiedene Verhaltensweisen getestet und optimiert 
werden können. Insgesamt bietet die rote Variante eine flexible, plattformunabhängige und gut analysierbare Simulationsumgebung mit modernen 
Webtechnologien und hoher Anpassungsfähigkeit.

\subsubsection*{Grüne Variante}
Die grüne Variante der Simulation ist ebenfalls plattformunabhängig und auf allen gängigen Betriebssystemen (Windows, Linux und MacOS) lauffähig. 
Die Darstellung erfolgt in 2D, was die Visualisierung der Simulation einfach und übersichtlich gestaltet. Für die Navigation und Pfadfindung 
wird der probabilistische Algorithmus D*Lite eingesetzt. Die Fahrzeugparameter sind fest und können während der Simulation nicht verändert 
werden, was zu einer stabilen und konsistenten Simulation führt. Der Datenexport erfolgt im JSON-Format, das universell und mit vielen 
Programmiersprachen kompatibel ist, was die Weiterverarbeitung und Integration in andere Systeme erleichtert. Die gesamte Simulation, 
sowohl das Frontend als auch das Backend, wird in Java entwickelt. Für das Frontend wird JavaFX verwendet, wodurch eine plattformübergreifende 
und benutzerfreundliche Desktop-Anwendung geschaffen wird. Java .jar-Dateien bieten eine einfache Möglichkeit, die Anwendung auf allen 
Betriebssystemen auszuführen. Zur Analyse der Simulationsergebnisse steht ein Dashboard zur Verfügung, das die verschiedenen Metriken 
und Daten visuell darstellt und eine tiefergehende Auswertung der Simulation ermöglicht. Die Simulationsgeschwindigkeit ist an Echtzeit 
orientiert, um ein realistisches Szenario zu schaffen, und kann passiv beobachtet werden, ohne dass direkt eingegriffen wird. Eine Besonderheit 
dieser Variante ist, dass die Fahrzeugparameter lernfähig sind. Das Fahrzeug kann seine Parameter im Laufe der Simulation automatisch optimieren 
und anpassen, um sich stetig zu verbessern. Insgesamt bietet die grüne Variante eine robuste, realistische und plattformübergreifende 
Simulationslösung mit einem leistungsfähigen Dashboard und adaptiven Fahrzeugparametern.

\end{document}
