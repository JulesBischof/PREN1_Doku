% \documentclass[DIV=21,12pt,captions=tableheading,landscape,a4paper]{scrreprt}
% \usepackage{booktabs}
% \usepackage{tikz} 

\usetikzlibrary{matrix, backgrounds} 

\newcommand\zeilenabstand{10pt} 

\def\GetNodeColumn(#1-#2-#3){#3}

\tikzset{vp/.style={circle, fill, inner sep=2pt}} 
\newcommand\verbindungslinie[4][10pt]{ 
  \node[#2, vp] at ([yshift=-.15*\zeilenabstand]#3.south) {};
  \edef\co{\noexpand\GetNodeColumn(#3)}
  \foreach[remember=\p as \lastp (initially #3)] \p in {#4} 
    \edef\cl{\noexpand\GetNodeColumn(\lastp)}
    \edef\cp{\noexpand\GetNodeColumn(\p)}
    \draw[ultra thick, #2, draw opacity=.4]
        ([yshift=-.15*\zeilenabstand,xshift={#1*(\co-\cl)}]\lastp.south) -- 
        ([yshift=-.15*\zeilenabstand,xshift={#1*(\co-\cp)}]\p.south) 
        node[vp] {}; 
} 


\documentclass[main.tex]{subfiles} % Subfile-Class

\begin{document}

\begin{landscape} % Querformat beginnen
  \thispagestyle{fancy}

  \section{Evaluation der Lösungsprinzipien}
  \subsection{Morphologischer Kasten}

  \begin{tikzpicture}
    \matrix[
      matrix of nodes,
      inner sep=0pt,
      row sep=\zeilenabstand,
      column sep=15pt,
      nodes={font=\strut},
      nodes in empty cells,
      column 2/.style={text width=4.5cm},
      column 3/.style={text width=3cm,align=center},
      column 4/.style={text width=3cm,align=center},
      column 5/.style={text width=3cm,align=center},
      column 6/.style={text width=3cm,align=center},
      column 7/.style={text width=3cm,align=center}
    ](m){
       & Detektieren eines Nodes
       & Phototransistor-Array
       & Kamera-Bilderkennung
       & Helligkeitssensor-Array
       & Distanz messen und abzählen
       & -                                                                                   \\
       & Detektieren einer Linie
       & Phototransistor-Array
       & Kamera-Bilderkennung
       & -
       & -
       & -                                                                                   \\
       & Unterschied Fliesenfuge, Klebeband
       & Rolltaster
       & zweiter Liniensensor
       & Bilderkennung
       & -
       & -                                                                                   \\
       & Erkennung fehlender Linien
       & Regelbasiert bestimmen (plausible Winkel)
       & KI-Modell
       & Probabilistisch (Auswertung Vergangenheit)
       & ohne
       & -                                                                                   \\
       & Erkennen von Hindernissen
       & gerichteter LIDAR
       & Bilderkennung
       & Lichtschranke
       & Taster
       & Ultraschall                                                                         \\
       & Erkennen von Pylonen
       & LIDAR
       & Bilderkennung
       & Lichtschranke
       & -
       & -                                                                                   \\
       & Unterschied Hindernis/Pylon
       & Höhendifferenz
       & Bilderkennung
       & -
       & -
       & -                                                                                   \\
       & Bestimmen aktueller Position
       & Gyroskop (Absolutwinkel zu Erdmagnetfeld)
       & Fahrwinkel und Distanz seit Start aufzeichnen
       & GPS
       & -
       & -                                                                                   \\
       & Kollisionen verhindern
       & Taster
       & Lichtschranke
       & LIDAR
       & Ultraschall
       & reine Vorplanung                                                                    \\
       & Controller
       & Gesamte Regelung auf HLC
       & Sensorik                                      
       & Aktorik über uC, Berechnung auf HLC
       & -
       & -                                                                                   \\
    };
    % Kopfzeile 
    \node(ul)[anchor=south west]
    at ([yshift={\zeilenabstand+\aboverulesep+\belowrulesep}]m.north west)
    {Parameter};
    \node(or)[anchor=south east] at (ul.north-|m-1-2.east){Ideen};
    \foreach[count=\i from 3] \l in {1,2,3,4,5}
    \node[anchor=base] at (or.base-|m-1-\i){\l};
    % Tabellenlinien 
    \draw[line width=\lightrulewidth](or.north-|ul.west)--(or.east|-ul.south)
    ([yshift=-\aboverulesep]ul.south-|m.west)
    --([yshift=-\aboverulesep]ul.south-|m.east);
    \draw[line width=\heavyrulewidth]([yshift=\belowrulesep]or.north-|m.west)
    --([yshift=\belowrulesep]or.north-|m.east)
    ([yshift={-\aboverulesep}]m.south west)
    --([yshift={-\aboverulesep}]m.south east);
    % Verbindungslinien 
    \begin{scope}[on background layer]
      \verbindungslinie{red}{m-1-3}{m-2-3,m-3-4,m-4-5,m-5-5,m-6-3,m-7-3,m-8-3,m-9-4,m-10-4}
      \verbindungslinie{blue}{m-1-4}{m-1-4,m-2-3,m-3-4,m-4-5,m-5-5,m-6-3,m-7-3,m-8-3,m-9-4,m-10-4}
      \verbindungslinie{green}{m-1-4}{m-2-4,m-3-5,m-4-6,m-5-4,m-6-4,m-7-4,m-8-4,m-9-4,m-10-3}
    \end{scope}
    % vertikale Beschriftung und Tabellenzwischenlinie
    \path(m-1-1.north west)--
    node[xshift=\zeilenabstand,rotate=90]{Erfassen der Umwelt}
    (m-9-1.south west);
  \end{tikzpicture}
  \captionof{table}{Morphologischer Kasten}\label{morphKasten}

  \newpage
  \begin{tikzpicture}
    \matrix[
      matrix of nodes,
      inner sep=0pt,
      row sep=\zeilenabstand,
      column sep=15pt,
      nodes={font=\strut},
      nodes in empty cells,
      column 2/.style={text width=3.75cm},
      column 3/.style={text width=3cm,align=center},
      column 4/.style={text width=3cm,align=center},
      column 5/.style={text width=3cm,align=center},
      column 6/.style={text width=3cm,align=center},
      column 7/.style={text width=3cm,align=center}
    ](m){
       & Weg durch Netz bestimmen
       & Wege gewichten
       & Wege vordefinieren
       & Brute force
       & bis zum nächsten Knoten vorausschauen
       & -                                         \\
       & Ansatz der Kartenerkennung
       & Rein auf auftretende Ereignisse reagieren
       & Bilderkennung und Route vorplanen
       & Nächste Knoten mit LIDAR vorplanen
       & Probabilistisch und Graph aufzeichnen
       & -                                         \\
       & Energiespeicher
       & Li-ion Akku
       & Ni-Cd Akku
       & Bleisäure Akku
       & Kraftstofftank
       & Druckluft                                 \\
       & Ladestands Anzeige
       & Mehrere LED's
       & 1 LED
       & Feedback WebUI
       & Analog
       & LCD 32x4                                  \\
       & Zielankunft signalisieren
       & Piezo-Alarm
       & Lautsprecher-Ton
       & Optische Anzeige (LED's)
       & LCD Display
       & -                                         \\
       & Zielerkennung
       & Bilderkennung (Beschriftung)
       & Reine Navigation
       & -
       & -
       & -                                         \\
    };
    % Kopfzeile 
    \node(ul)[anchor=south west]
    at ([yshift={\zeilenabstand+\aboverulesep+\belowrulesep}]m.north west)
    {Parameter};
    \node(or)[anchor=south east] at (ul.north-|m-1-2.east){Ideen};
    \foreach[count=\i from 3] \l in {1,2,3,4,5}
    \node[anchor=base] at (or.base-|m-1-\i){\l};
    % Tabellenlinien 
    \draw[line width=\lightrulewidth](or.north-|ul.west)--(or.east|-ul.south)
    ([yshift=-\aboverulesep]ul.south-|m.west)
    --([yshift=-\aboverulesep]ul.south-|m.east);
    \draw[line width=\heavyrulewidth]([yshift=\belowrulesep]or.north-|m.west)
    --([yshift=\belowrulesep]or.north-|m.east)
    ([yshift={-\aboverulesep}]m.south west)
    --([yshift={-\aboverulesep}]m.south east);
    % Verbindungslinien 
    \begin{scope}[on background layer]
      \verbindungslinie{red}{m-1-6}{m-2-6,m-3-3,m-4-7,m-5-3,m-6-3}
      \verbindungslinie{blue}{m-1-3}{m-2-6,m-3-3,m-4-7,m-5-3,m-6-4}
      \verbindungslinie{green}{m-1-5}{m-2-3,m-3-4,m-4-3,m-5-4,m-6-3}
    \end{scope}
    % vertikale Beschriftung und Tabellenzwischenlinie
    \path(m-1-1.north west)--
    node[xshift=\zeilenabstand,rotate=90]{Navigation}
    (m-2-1.south west);
    \draw[line width=\lightrulewidth]([yshift=-\tabcolsep]m.west|-m-2-3.south)--([yshift=-\tabcolsep]m.east|-m-2-3.south);
    \path(m-3-1.north west)--
    node[xshift=\zeilenabstand,rotate=90]{Energie}
    (m-4-1.south west);
    \draw[line width=\lightrulewidth]([yshift=-\tabcolsep]m.west|-m-4-3.south)--([yshift=-\tabcolsep]m.east|-m-4-3.south);
    \path(m-5-1.north west)--
    node[xshift=\zeilenabstand,rotate=90]{Ziel}
    (m-6-1.south west);
  \end{tikzpicture}
  \captionof{table}{Morphologischer Kasten}\label{morphKasten}

  \newpage
  \begin{tikzpicture}
    \matrix[
      matrix of nodes,
      inner sep=0pt,
      row sep=\zeilenabstand,
      column sep=15pt,
      nodes={font=\strut},
      nodes in empty cells,
      column 2/.style={text width=3.75cm},
      column 3/.style={text width=3cm,align=center},
      column 4/.style={text width=3cm,align=center},
      column 5/.style={text width=3cm,align=center},
      column 6/.style={text width=3cm,align=center},
      column 7/.style={text width=3cm,align=center}
    ](m){
       & Fortbewegungsmittel
       & Gummiräder
       & Mecanumräder
       & 4 Beine (gehend)
       & Propeller (fliegend)
       & Gummiraupen                                 \\
       & Antrieb
       & DC-Motor
       & Dampfmaschine
       & Schrittmotor
       & Stirlingmotor
       & Verbrennungsmotor                           \\
       & Fahrwerk Lenkung
       & 2 Rädrig (Achse mittig) mit 2 Kugeln
       & 2 Rädrig mit einer Kugel
       & 3 Rädrig mit lenkendem Einzelrad
       & 4 Rädrig mit lenkender Achse
       & Panzerlenkung                               \\
       & Hindernis Aufnahmevorrichtung
       & von oben ohne Winkelausrichtung
       & von oben mit Winkelausrichtung
       & von oben frei rotierend und Arretierung
       & Hindernis seitlich aufnehmen
       & durch die mittleren Löcher aufnehmen        \\
       & Hindernis aufnehmen
       & Gabelstapler
       & Klemmgreifer
       & Saugnapf
       & Schnappverschluss
       & -                                           \\
       & Hindernis-Handling
       & Über das Fahrzeug drüber
       & Um das Fahrzeug herum
       & Linearförderung durch das Fahrzeug
       & Fahrzeug 360° Drehung
       & Fahrzeug 180° Drehung (symmetrisch)         \\
       & Hindernis positionieren
       & anhand gefahrender Distanz und Liniensensor
       & per Kamera
       & -
       & -
       & -                                           \\
       & Z-Positionierung
       & Roboterarm
       & Linearmodul
       & Hebebühne
       & Pneumatik
       & Hubmagnet                                   \\
       & Notstopp
       & Buzzer
       & Taster
       & -
       & -
       & -                                           \\
    };
    % Kopfzeile 
    \node(ul)[anchor=south west]
    at ([yshift={\zeilenabstand+\aboverulesep+\belowrulesep}]m.north west)
    {Parameter};
    \node(or)[anchor=south east] at (ul.north-|m-1-2.east){Ideen};
    \foreach[count=\i from 3] \l in {1,2,3,4,5}
    \node[anchor=base] at (or.base-|m-1-\i){\l};
    % Tabellenlinien 
    \draw[line width=\lightrulewidth](or.north-|ul.west)--(or.east|-ul.south)
    ([yshift=-\aboverulesep]ul.south-|m.west)
    --([yshift=-\aboverulesep]ul.south-|m.east);
    \draw[line width=\heavyrulewidth]([yshift=\belowrulesep]or.north-|m.west)
    --([yshift=\belowrulesep]or.north-|m.east)
    ([yshift={-\aboverulesep}]m.south west)
    --([yshift={-\aboverulesep}]m.south east);
    % Verbindungslinien 
    \begin{scope}[on background layer]
      \verbindungslinie{red}{m-1-3}{m-2-5,m-3-4,m-4-3,m-5-4,m-6-6,m-7-3,m-8-4,m-9-3}
      \verbindungslinie{blue}{m-1-3}{m-2-3,m-3-3,m-4-7,m-5-3,m-6-7,m-7-3,m-8-4,m-9-3}
      \verbindungslinie{green}{m-1-4}{m-2-3,m-3-7,m-4-5,m-5-4,m-6-4,m-7-4,m-8-5,m-9-3}
    \end{scope}
    % vertikale Beschriftung und Tabellenzwischenlinie
    \path(m-1-1.north west)--
    node[xshift=\zeilenabstand,rotate=90]{Chassis}
    (m-9-1.south west);
  \end{tikzpicture}
  \captionof{table}{Morphologischer Kasten}\label{morphKasten}

\end{landscape} % Querformat beenden

\end{document}