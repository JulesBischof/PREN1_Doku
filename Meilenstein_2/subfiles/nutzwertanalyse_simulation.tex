\documentclass[main.tex]{subfiles} % Subfile-Class


% ============================================================================== %
%                            Subfile document                                    %
% ============================================================================== %

\begin{document}

% Template

\subsection{Nutzwertanalyse Simulation}

Wie man an der Nutzwertanalyse auf der nächsten Seite erkennen kann, entschieden wir
uns für ein web-basierte Lösung, in welchem wir die Simulation top-down in einem 2d-Format darstellen.
Hauptgrund für das Wählen einer web-basierten Lösung war die plattformübergreifende Verfügbarkeit einer
solchen Software, da wir im Team mit unterschiedlichen Plattformen arbeiten. Dadurch lässt sich
nicht-funktionalen Aufwand bei der Entwicklung sparen, der nur dafür gebraucht worden wäre um die Simulation
bei allen aufzusetzen. Zusätzlich ist jedes Teammitglied mit Weboberflächen bestens vertraut und somit in
der Lage den Simulator ohne grosse Einarbeitungszeit zu bedienen.
\\\\
Wir waren uns bereits früh in der Planung des Simulators einig, dass wir
die Simulation nur in zwei Dimensionen darstellen wollen, da wir keinen Mehrwert in einer
dreidimensionalen Darstellung sahen. Der verhältnismässig hohe Entwicklungsaufwand für keinen erkennbaren
Mehrwert war der ausschlaggebende Grund für unsere Entscheidung. Da wir den Simulator hauptsächlich als
Testumgebung für unsere Pfadfindungsalgorithmen sehen, ist die die realitätsnahe Darstellung dder Simulation
nicht von grosser Bedeutung. Die Bedienbarkeit der Simulation ist jedoch ein wichtiger Faktor, da wir
die zu implementierenden Algorithmen ausgiebig testen wollen und somit viel Zeit mit der Bedienung verbracht wird.

\begin{table}[ht]
    \centering
    \begin{tabular}{|p{0.14\linewidth}|p{0.15\linewidth}|p{0.115\linewidth}|p{0.08\linewidth}|p{0.09\linewidth}|p{0.08\linewidth}|p{0.09\linewidth}|}
    \hline
        &&&&&&\\[-9pt]
        & \multirow{2}{*}{\textbf{Kriterien}} & Gewichtung & Punkte & Punkte & Punkte & Punkte \\[1pt]
        &  & \{\%\} & \{1...10\} & gewichtet & \{1...10\} & gewichtet \\[1pt]
        \hline
        \hline
        & \multicolumn{2}{c|}{} & \multicolumn{2}{c|}{} & \multicolumn{2}{c|}{} \\[-9pt]
        \multirow{5}{4em}{\textbf{Plattform}} & \multicolumn{2}{c|}{} & \multicolumn{2}{c|}{\textbf{Nativ}} & \multicolumn{2}{c|}{\textbf{Web-based}} \\[1pt]
        \cline{2-7}
        &&&&&&\\[-9pt]
        & Entwicklungs-aufwand & 40 & 2 & 8 & 8 & 32 \\[1pt]
        \cline{2-7}
        &&&&&&\\[-9pt]
        & Performance & 20 & 7 & 14 & 4 & 8 \\[1pt]
        \cline{2-7}
        &&&&&&\\[-9pt]
        & Bedienbarkeit & 40 & 4 & 16 & 7 & 28 \\[1pt]
        \cline{2-7}
        &&&&&&\\[-9pt]
        & \textbf{Summe} & \textbf{100} &  & \textbf{38} &  & \textbf{68} \\[1pt]
        \hline
        \hline
        & \multicolumn{2}{c|}{} & \multicolumn{2}{c|}{} & \multicolumn{2}{c|}{} \\[-9pt]
        \multirow{6}{4em}{\textbf{Visualisier-ung}} & \multicolumn{2}{c|}{} & \multicolumn{2}{c|}{\textbf{3D}} & \multicolumn{2}{c|}{\textbf{2D}} \\[1pt]
        \cline{2-7}
        &&&&&&\\[-9pt]
        & Entwicklungs-aufwand & 40 & 3 & 12 & 8 & 32 \\[1pt]
        \cline{2-7}
        &&&&&&\\[-9pt]
        & Interaktions-möglichkeit & 10 & 6 & 6 & 5 & 5 \\[1pt]
        \cline{2-7}
        &&&&&&\\[-9pt]
        & Realitätsnähe & 20 & 8 & 16 & 3 & 6 \\[1pt]
        \cline{2-7}
        &&&&&&\\[-9pt]
        & Bedienbarkeit & 30 & 5 & 15 & 5 & 15 \\[1pt]
        \cline{2-7}
        &&&&&&\\[-9pt]
        & \textbf{Summe} & \textbf{100} &  & \textbf{49} &  & \textbf{58} \\[1pt]
        \hline
        \hline
        & \multicolumn{2}{c|}{} & \multicolumn{2}{c|}{} & \multicolumn{2}{c|}{} \\[-9pt]
        \multirow{5}{4em}{\textbf{Auswertung}} & \multicolumn{2}{c|}{} & \multicolumn{2}{c|}{\textbf{In-tool}} & \multicolumn{2}{c|}{\textbf{Export + Scripts}} \\[1pt]
        \cline{2-7}
        &&&&&&\\[-9pt]
        & Entwicklungs-aufwand & 30 & 5 & 15 & 5 & 15 \\[1pt]
        \cline{2-7}
        &&&&&&\\[-9pt]
        & Flexibilität & 40 & 4 & 16 & 6 & 24 \\[1pt]
        \cline{2-7}
        &&&&&&\\[-9pt]
        & Integration in Dokumentation & 30 & 2 & 6 & 8 & 24 \\[1pt]
        \cline{2-7}
        &&&&&&\\[-9pt]
        & \textbf{Summe} & \textbf{100} &  & \textbf{37} &  & \textbf{63} \\[1pt]
        \hline
        \hline
        & \multicolumn{2}{c|}{} & \multicolumn{2}{c|}{} & \multicolumn{2}{c|}{} \\[-9pt]
        \multirow{6}{4em}{\textbf{Wieder-verwendbar-keit}} & \multicolumn{2}{c|}{} & \multicolumn{2}{c|}{\textbf{Splitting}} & \multicolumn{2}{c|}{\textbf{Kein Splitting}} \\[1pt]
        & \multicolumn{2}{c|}{} & \multicolumn{2}{c|}{\textbf{Front-/Backend}} & \multicolumn{2}{c|}{\textbf{}} \\[1pt]
        \cline{2-7}
        &&&&&&\\[-9pt]
        & Entwicklungs-aufwand & 40 & 3 & 12 & 7 & 28 \\[1pt]
        \cline{2-7}
        &&&&&&\\[-9pt]
        & Austauschbar-keit & 30 & 8 & 24 & 2 & 6 \\[1pt]
        \cline{2-7}
        &&&&&&\\[-9pt]
        & Transpilation & 30 & 8 & 24 & 2 & 6 \\[1pt]
        \cline{2-7}
        &&&&&&\\[-9pt]
        & \textbf{Summe} & \textbf{100} &  & \textbf{60} &  & \textbf{40} \\[1pt]
        \hline
        \hline
        \multicolumn{2}{|c|}{} &&\multicolumn{2}{c|}{}&\multicolumn{2}{c|}{} \\[-9pt]
        \multicolumn{2}{|c|}{\textbf{Gesamtsumme}} &  & \multicolumn{2}{c|}{\textbf{46}} & \multicolumn{2}{c|}{\textbf{57.25}} \\[1pt]
        \hline
        \end{tabular}
    \caption{Nutzwertanalyse Simulation}
\end{table}

\end{document}
