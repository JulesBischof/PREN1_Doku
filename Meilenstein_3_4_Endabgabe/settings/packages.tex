% packages.tex

% ---------------- used packages for LaTeX Document

\usepackage[english, german]{babel}                 % Language support for English and German. Handles language-specific typographic rules and hyphenation.
\usepackage{listings}                               % Provides support for including source code in your document. Customizable for various programming languages.
\usepackage{caption}                                % customised captions in floating environments
\usepackage{xcolor}                                 % Provides color support. Allows for color definitions and styling, including colored text, tables, and code.
\usepackage{array}                                  % Enhances the array and tabular environments with additional features like column definitions and customizations.
\usepackage{graphicx}                               % Improves handling of graphics. Allows for easy inclusion, scaling, and rotation of images.
\usepackage{tabularx}                               % Provides an extended tabular environment with more flexibility in defining table widths and column types.
\usepackage{booktabs}                               % For professional looking tables
\usepackage{amsmath}                                % Provides enhanced mathematical typesetting. Includes additional symbols, environments, and improved equation formatting.
\usepackage{tcolorbox}                              % Provides colored boxes for creating styled content blocks like notes, warnings, and examples.
\usepackage{fontawesome5}                           % Provides access to Font Awesome 5 icons for use in documents. Useful for adding symbols and icons to content.
\usepackage{pdfpages}                               % Allows for including external PDF files in the document. Useful for adding full-page PDFs, like appendices or reports.
\usepackage{pdflscape}                              % Provides landscape pages for PDF output. Allows individual pages to be rotated to landscape orientation.
\usepackage{subfiles}                               % Facilitates the use of subfiles in large documents. Allows for separate LaTeX files for different sections of a project.
\usepackage{csquotes}                               % Provides enhanced quote handling with support for different quotation styles and languages. Ensures correct typographic quotation marks.
\usepackage{pdflscape}                              % Provides landscape pages for PDF output. Allows individual pages to be rotated to landscape orientation.
\usepackage{lipsum}                                 % Generates placeholder text for testing. Useful for creating sample text for documents and templates.
\usepackage{hyperref}                               % Provides support for hyperlinks in the document. Allows for clickable links for references, URLs, and internal document links.
\usepackage{longtable}                              % Allows for tables that span multiple pages. Useful for large tables that do not fit on a single page.
\usepackage[backend=biber, style=ieee]{biblatex}    % Bibliography management. Uses Biber as the backend for processing bibliographic data and formats citations and references in IEEE style.
\usepackage{fancyhdr}                               % Headers
\usepackage{float}                                  % Improves the interface for defining floating objects such as figures and tables.
\usepackage{longtable}                              % larger tabular environment
\usepackage{hyperref}                               % Provides support for hyperlinks in the document. Allows for clickable links for references, URLs, and internal document links.
\usepackage{xr}                                     % references to external LaTex Docs
\usepackage{xr-hyper}
\usepackage{multirow}
\usepackage{tikz}
\usepackage{makecell}
\usepackage[table]{xcolor}                          % Colors in tables
\usepackage{amsmath}                                % Align and mathematical environment
\usepackage{siunitx}                                % SI units
