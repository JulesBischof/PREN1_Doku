\documentclass[main.tex]{subfiles} % Subfile-Class


% ============================================================================== %
%                            Subfile document                                    %
% ============================================================================== %

\begin{document}

\section*{Abstract}

Im interdisziplinären Projektmodul PREN 1 der Hochschule Luzern wurde ein Konzept 
für einen autonomen Roboter entwickelt, der ein vorgegebenes Wegenetzwerk selbstständig 
navigieren kann. Ziel war es, Hindernisse zu 
erkennen, gesperrte Wegpunkte zu meiden und dynamisch die effizienteste Route vom 
Start- zum Zielpunkt zu ermitteln. Der Roboter nutzt Sensordaten von LIDAR, Kamera 
und Ultraschallsensoren, um eine präzise Umgebungserkennung zu gewährleisten, und 
basiert auf einem heuristikbasierten Algorithmus, der flexibel auf unbekannte 
Graphen reagieren kann.\\

Besonderer Fokus lag auf der Konstruktion eines stabilen, gewichtsoptimierten Chassis, 
das den strengen Vorgaben des Projekts entspricht, sowie auf der Entwicklung eines 
motorisierten Greifers zur aktiven Hindernisbeseitigung. Zur Unterstützung der 
Konzeptentwicklung wurde ein Simulator implementiert, der eine frühzeitige Validierung 
der Navigationsalgorithmen in einer virtuellen Umgebung erlaubt. 
Nachhaltigkeitsaspekte wie energieeffiziente Komponenten, ressourcenschonende 
Materialwahl und der Bezug zu den Zielen für nachhaltige Entwicklung (SDGs) wurden 
systematisch berücksichtigt.\\

Die im Modul PREN 1 erarbeiteten Ergebnisse bilden die Grundlage für die praktische 
Umsetzung und den Bau des Roboters im nachfolgenden Modul PREN 2. Dort wird das 
Konzept weiter optimiert und die technische Machbarkeit unter realen Bedingungen 
evaluiert, bevor ein funktionales Gesamtsystem präsentiert wird.

\end{document}

