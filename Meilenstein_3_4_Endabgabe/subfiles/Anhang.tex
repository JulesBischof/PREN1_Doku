\documentclass[main.tex]{subfiles} % Subfile-Class

% ============================================================================== %
%                            Anhang document                                    %
% ============================================================================== %

\begin{document}

\section{Anhang}
\appendix

\subfile{./Anhang_subfiles/Evaluation_Mechanik_subfiles/Greifereinheit.tex}
\newpage

\subfile{./Anhang_subfiles/Evaluation_Elektronik_subfiles/elektro_gesamtuebersicht.tex}
\newpage

\subfile{./Anhang_subfiles/Evaluation_Elektronik_subfiles/Boardnetz.tex}
\newpage

\subfile{./Anhang_subfiles/Evaluation_Elektronik_subfiles/Antriebe.tex}
\newpage

\subfile{./Anhang_subfiles/Evaluation_Elektronik_subfiles/Hinderniserkennung.tex}
\newpage

\subfile{./Anhang_subfiles/Evaluation_Elektronik_subfiles/Strecke_Tracken.tex}
\newpage

\subfile{./Anhang_subfiles/Evaluation_Elektronik_subfiles/Liniensensor.tex}
\newpage

\subfile{./Anhang_subfiles/Projektmanagement_subfiles/Risikomanagement.tex}
\newpage

\subfile{./Anhang_subfiles/Projektmanagement_subfiles/Recherche.tex}
\newpage

\subfile{./Anhang_subfiles/Projektmanagement_subfiles/Anforderungsliste.tex}
\newpage

\subfile{./Anhang_subfiles/Projektmanagement_subfiles/Morphologischer_Kasten.tex}
\newpage

\section{Digitaler Anhang}
Dem Abgabedokument beigefügt ist ein digitaler Anhang. Darin befinden sich
Schema und Layout Daten zu den verschiedenen Leiterplatten sowie
Konstruktionsdaten und CAD-Modelle des aktuellen Arbeitsstands.

Das Verzeichnis ist der README Datei zu entnehmen, welche sich im Anhang
befindet.

\end{document}
