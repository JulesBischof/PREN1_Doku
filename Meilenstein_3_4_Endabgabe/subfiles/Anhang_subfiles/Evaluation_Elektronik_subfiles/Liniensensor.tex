\documentclass[main.tex]{subfiles} % Subfile-Class


% ============================================================================== %
%                            Subfile document                                    %
% ============================================================================== %

\begin{document}

% Template

\section{Dimensionierung Liniensensor mit UV und IR Lichtspektren}

Im folgendem Abschnitt wird untersucht ob sich das infrarot Spektrum oder das ultraviolette Spektrum besser für
die Unterscheidung des Klebebands zum Wettkampfuntergrund eignet. Zuerst müssen aber alle Komponenten der Messzellen, 
welche mit unterschiedlichen Wellenlängen emittierenden Dioden und Fototransistoren ausgestattet sind, berechnet werden.


\subsection{Dimensionierung IR-Messzelle}
In diesem Abschnitt wird eine IR-Messzelle dimensionert. Gemäss Datenblatt kann über dem IR-Emitter von 1.3 bis 2V
abfallen. Die Speisespannung beträgt 3.3V. Der maximaler Strom beträgt 100mA. Der Strom wird gewählt, dass er ungefähr im 
Bereich von 10mA bis 60mA fällt. Dies wurde so gewählt, das nicht zu viel Leistung verbraucht wird, aber der Emitter trotzdem 
mit genügen Strom versorgt wird. Daraus resultiert die folgende Widersandsberechnung:
\[
    R_{max} = \frac{U_q - U_{LED}}{I_{LEDmin}} = \frac{3.3V - 1.3V}{0.01A} = 200ohm
\]
\[
    R_{min} = \frac{U_q - U_{LED}}{I_{LEDmax}} = \frac{3.3V - 1.3V}{0.03A} = 33.33ohm
\]
Daraus resultiert ein Potentiometer Rp1 von 200ohm und ein R1.1 von 33ohm.

Nun wird der Widerstand R1.3 Dimensioniert. Die Speisespannung von 3.3V wird direkt von einem Microcontoller übernommen
und der Fototransistor wird als Konstantstromquelle betrachtet. Der daraus resultierende Strom wurde in einem Versuch gemessen
und im AnhangXYZ dokumentiert. Daher resultiert der Widerstand:... !!!!!Werte nachfragen!!!!!!

%===========================================================================================================%

\subsection{Dimensionierung UV-Messzelle}
In diesem Abschnitt wird die UV-Messzelle dimensioniert.
Über die Speisespannung von 5V wird eine UV-Emitter mit Vorwiderstand bestromt. Der empfohlene Strom
wird Gemäss dem Datenblatt des UV-Emitters von 10mA bis 20mA vorgeschlagen. Ausserdem darf der Strom nicht 
30mA übersteigen. Daher resultiert: 

\[
    R_{max} = \frac{U_q - U_{LED}}{I_{LEDmin}} = \frac{5V - 2.9V}{0.01A} = 210ohm
\]
\[
    R_{min} = \frac{U_q - U_{LED}}{I_{LEDmax}} = \frac{5V - 2.9V}{0.03A} = 70ohm
\]
Daraus resultiert ein Potentiometer Rp1 von 200ohm und ein R1.1 von 82ohm, welches ungefähr im zuvor festgelegten
Strombereich liegt:
\[
    I_{max} = \frac{U_R}{R_{min}} = \frac{2.1}{82ohm} = 0.0256A = 25.6mA
\]
\[
    I_{min} = \frac{U_R}{R_{max}} = \frac{2.1}{282ohm} = 0.00745A = 7.45mA
\]
Aufgrund dieser Widerstandsaufteilung kann der Strom ungefähr in dem empfohlenen Bereich frei eingestellt werden.


Nun wird der Widerstand R1.3 Dimensioniert. Die Speisespannung von 3.3V wird direkt von einem Microcontoller übernommen
und der Fototransistor wird als Konstantstromquelle betrachtet. Der daraus resultierende Strom wurde in einem Versuch gemessen
und im AnhangXYZ dokumentiert. Daher resultiert der Widerstand:... !!!!!Werte nachfragen!!!!!!

\subsection{Versuchsmessungen}
!!!!!!Messungen aufzeigen und vergleichen!!!!!

%===========================================================================================================%
\end{document}
