\documentclass[main.tex]{subfiles} % Subfile-Class


% ============================================================================== %
%                            Subfile document                                    %
% ============================================================================== %

\begin{document}

% ============================================================================== %
\section{Konzepterstellung Greifereinheit}~\label{appendix:Greifereinheit}

In diesem Abschntt wird die Entwicklung des Greifarmkonzepts behandelt.
Zunächst werden die benötigten Anforderungen erfasst. 
Danach werden die verschiedenen Konzepte kommentiert.
Schliesslich wird eine Entscheidung auf Grundlage der Testergebnisse getroffen und dokumentiert.

% ============================================================================== %
\subsection*{Anforderungen}

\paragraph{Greifkraft}
Die Greifkraft muss ausreichend dimensioniert sein, um das Hindernis sicher greifen zu können.
Dabei sind die Haftreibung und die Anpresskraft zu berücksichtigen.
\[
    F_{erforderlich} = \frac{m \cdot g}{\mu_{\text{hr}}}
\]

\[
    M_{erforderlich} = F_{erforderlich} \cdot Hebel \cdot Sicherheit
\]
Das erforderliche Drehmoment beträgt $0.235 Nm$.

\paragraph{Höhenverstellung}
Die Höhenverstellung muss an das Gewicht des zu hebenden Hindernisses, des Greifers und der Elektronik ausgelegt sein.
\[
    M_{erforderlich} = (m_{Hindernis} + m_{Elektronik} + m{Greifer}) \cdot Hebel \cdot Sicherheit
\]
Das erforderliche Drehmoment beträgt $0.2 Nm$.

\paragraph{Genauigkeit}
Der Greifer muss den gesamten Bewegungsablauf mit hoher Wiederholgenauigkeit ausführen.
Dadurch wird sichergestellt, dass das Hindernis innerhalb des Toleranzbereichs
zurückplatziert wird.

\paragraph{Gewicht}
Das Gewicht der Greifereinheit ist gering zu halten, um das Gesamtgewicht des Fahrzeugs zu
reduzieren. Eine Gewichtsreduktion ist ebenfalls bei der bewegten Massen des Greifers wichtig, da sie
direkte Auswirkungen auf die Motorauswahl, die Geschwindigkeit und die Energieeffizient hat.

\newpage

% ============================================================================== %
\subsection*{Konzeption}

\subsubsection*{Konzept 2 - Greifer und Höhenverstellung - ein Motor}

Bild \newline

In diesem Konzept wird ein einzelner Motor sowohl für die Greiffunktion als auch für die Höhenverstellung eingesetzt. 
Die Höhenverstellung erfolgt mit einem Zahnrad, welches von einen Motor angetrieben wird und einer Zahnstange, die
sich vertikal nach oben oder unten bewegt.

Der Greifer besteht aus zwei Backen: die vordere Backe ist fest montiert, während die hintere Backe beweglich 
auf einer Gleitführung sitzt. Die Backen sind mit einem Gummiband vorgespannt, sodass sie im Ruhezustand 
geschlossen bleiben. An den Seiten der hinteren Backe befinden sich Mitnehmer, die an einer linearen Nocke 
auf beiden Seiten geführt wird. Diese Nocken bewegt die hintere Backe an die gewünschte Position.
Dies ermöglicht die Öffnung und Schliessung des Greifers \newline


Bild lineare Nocke (evtl.Bewegungsablauf erklären und noch Klappe) \newline
Hier ist ein Bewegungsablauf... Klappe für verschiedenen Weg...


\subsubsection*{Konzept 1 - Greifer und Höhenverstellung - zwei Motoren}

Bild \newline

In diesem Konzept werden jeweils separate Motoren für die Greiffunktion und für die Höhenverstellung eingesetzt. 
Der Greifer wird als Parallelgreifer umgesetzt. Ein Motor, der mit einem Zahnrad gekoppelt ist, treibt zwei 
Zahnstangen an, wodurch sich die beiden Greiferbacken horizontal bewegen.

Die Höhenverstellung erfolgt über die Ausgangswelle eines weiteren Motors, die fest mit einem Zahnrad verbunden ist.
Dieses Zahnrad treibt eine Zahnstange an, die eine vertikale Bewegung nach oben oder unten ermöglicht.


\subsubsection*{Konzept 3 - Gabelstapler - ein Motor}

Bild \newline

In diesem Konzept wird das Prinzip eines Gabelstaplers nachgebildet. Die Gabeln werden in die dafür vorgesehene 
Öffnung des Hindernisses eingeführt. Anschließend wird das Hindernis mithilfe eines Motors vertikal angehoben, 
wobei die Hebebewegung durch ein Zahnstangen-Zahnrad-System realisiert wird.

\end{document}

\newpage

\subsubsection*{Fazit und Entscheidung der Konzeptphase}

\subsection*{Entwicklung und Dimensionierung}

\paragraph{Konzept 1}
Probleme
Lösungen
Weitere Probleme etc.

\paragraph{Konzept 2}
Probleme
Lösungen
Weitere Probleme etc.

\subsection*{Versuche}

\paragraph{Konzept 1}
Versuch 1, lineare Nocke zu wenig steil etc.
Versuch 2, Linearführung gleitet zu wenig
Versuch 3, mit Motor

\paragraph{Konzept 2}
Versuch 1, mit kleinem Servo -> zu schwach
Versuch 2, \dots

\subsection{Fazit und Ausblick}


