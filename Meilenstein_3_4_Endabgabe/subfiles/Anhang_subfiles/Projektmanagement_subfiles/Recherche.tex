\documentclass[main.tex]{subfiles} % Subfile-Class

\usepackage{array}
\usepackage{lipsum}

% ============================================================================== %
%                            Subfile document                                    %
% ============================================================================== %

\begin{document}

% Template

\section{Recherche}~\label{appendix:Recherche}

\noindent
Die nachfolgende Quellensammlung in \autoref{tab:sources} dient als Übersicht
zur Technologierecherche und wird im Laufe des Projekts
weitergeführt, um die verwendeten Quellen im Ausblick auf die
Schlussdokumentation zu sammeln. Die unter Grade aufgeführten Werte
dienen zur Bewertung der Relevanz der Quellen für das Projekt und
deren weiterführende Benutzung.

\subsection{Quellen}

\renewcommand{\arraystretch}{1.3}
\begin{longtable}{@{} L{2.5cm} L{2.9cm} C{1cm} L{1.2cm} L{6cm} @{}}

% Table header
\toprule
\textbf{Thema} & \textbf{Stichwort} & \textbf{Grade} & \textbf{Quelle} & \textbf{Beschreibung} \\
\midrule
\endfirsthead

% Table header for subsequent pages
\multicolumn{5}{c}%
{{\bfseries \tablename\ \thetable{} -- Fortsetzung}} \\
\toprule
\textbf{Thema} & \textbf{Stichwort} & \textbf{Grade} & \textbf{Quelle} & \textbf{Beschreibung} \\
\midrule
\endhead

% Footer for all pages except the last
\midrule
\multicolumn{5}{r}{{contd}} \\
\caption{Quellensammlung}\label{tab:sources} \\
\endfoot

% Footer for the last page
\bottomrule
\caption{Quellensammlung}\label{tab:sources} \\
\endlastfoot

Simulator & Pfadfindung & 2 & \href{https://clementmihailescu.github.io/Pathfinding-Visualizer/\#}{Link} & Visualisierung verschiedener Pfadfindungsalgorithmen. \\
\hline
Simulator & Pfadfindung & 5 & \href{https://scholar.uwindsor.ca/cgi/viewcontent.cgi?article=9230\&context=etd}{Link} & Performance Evaluation von Pfadfindungsalgorithmen. \\
\hline
Simulator & Graph & 3 & \href{https://visjs.github.io/vis-network/examples/}{Link} & Erstellung von 2D Graphen. \\
\hline
Simulator & 2D-Simulation für autonome Fahrzeuge & 4 & \href{https://docs.unity3d.com/Simulation/manual/author/create-a-vehicle-model.html}{Link} & Simulationstool für Visualisierung. \\
\hline
Simulator & Sensoren und KI & 4 & \href{https://medium.com/@krish.bhoopati556/coding-the-road-ahead-self-driving-cars-with-javascript-and-ai-1995ecb2c1ec}{Link} & Programmierung von Sensoren und neuronalen Netzen in Javascript. \\
\hline
Simulator & Physik Auto & 4 & \href{https://www.asawicki.info/Mirror/Car\%20Physics\%20for\%20Games/Car\%20Physics\%20for\%20Games.html}{Link} & Simulation eines realistischen Fahrverhaltens. \\
\hline
Simulator & Editierbare Benutzeroberflächen & 5 & \href{https://github.com/tpan496/csgraph\_editor}{Link} & Benutzerfreundliche Oberfläche. \\
\hline
Simulator & Pfadfindung, Berechenbarkeit & 8 & \href{https://moribots.github.io/project/motion-planning}{Link} & Übersicht und Visualisierung verschiedener fortgeschrittener Pfadfindungsalgorithmen. \\
\hline
Simulator & Pfadfindung & 6 & \href{https://www.mcrl2.org/web/\_downloads/08c5ec296da673df304284f04f0e815a/state-space-exploration.pdf}{Link} & State Space Exploration: Grundlagen der Graphenexploration. \\
\hline
Simulator & Pfadfindung & 5 & \href{https://sites.gatech.edu/acds/mppi/}{Link} & Übersicht über Model Predictive Path Integral (MPPI). \\

Simulator & Pfadfindung, Optimierungen & 7 & \href{https://ai.dmi.unibas.ch/\_files/teaching/hs21/po/slides/po-f01.pdf}{Link 1}, \href{https://www.math.leidenuniv.nl/\~{}kallenberglcm/Lecture-notes-MDP.pdf}{Link 2} & Markov Decision Processes (MDP): Modellierungen von Entscheidungen bei ungewissem Ausgang, welcher Weg ist wahrscheinlich der schnellste im Graph. \\
\hline
Simulator & Pfadfindung & 7 & \href{https://idm-lab.org/bib/abstracts/papers/aaai02b.pdf}{Link} & Detaillierte Beschreibung des D\textsuperscript{*}Lite Algorithmus. \\
\hline
Simulator & Pfadfindung & 4 & \href{http://sibgrapi.sid.inpe.br/col/sid.inpe.br/banon/2002/11.13.11.53/doc/269-275.pdf}{Link} & Euclidean Distance Transform für heuristische Entscheidungen bei Graphenproblemen. \\
\hline
Sensorik & Raumwahrnehmung, Image Processing & 5 & \href{http://www.robotics.stanford.edu/\~{}ang/papers/aaai08-Make3dDepthPerceptionSingleImage.pdf}{Link} & Depth Perception: Grundlagen für Raumwahrnehmung bei der Bildverarbeitung. \\
\hline
Sensorik & Homographie, Image Processing & 5 & \href{https://docs.opencv.org/4.x/d9/dab/tutorial\_homography.html}{Link} & Informationen, um verzerrte Bilder in verschiedene Perspektiven zu transformieren. \\
\hline
Sensorik & Kantenerkennung, Image Processing & 9 & \href{https://docs.opencv.org/4.x/da/d22/tutorial\_py\_canny.html}{Link 1}, \href{https://iopscience.iop.org/article/10.1088/1757-899X/1096/1/012079/pdf}{Link 2} & Erkennung von Kanten in Bildern, ermöglicht rudimentäre Kollisionserkennung. \\
\hline
Sensorik & Image Processing & 8 & \href{https://hal.science/hal-04132827/document}{Link} & Analyse von mehreren SLAM Algorithmen. \\
\hline
Elektrotechnik - Antriebe & BLDC Grundlagen & 10 & \href{https://www.microchip.com/en-us/application-notes/an885}{Link} & Application Note: Grundlagen BLDC Motoren. \\
\hline
Elektrotechnik - Antriebe & BLDC Grundlagen & 6 & \href{https://www.silabs.com/documents/public/application-notes/an0816-efm32-brushless-dc-motor-control.pdf}{Link} & Application Note: Grundlagen BLDC Motoren. \\
\hline
Elektrotechnik - Antriebe & Brushless DC Motor Fundamentals & 7 & \href{https://media.monolithicpower.com/document/Brushless\_DC\_Motor\_Fundamentals.pdf}{Link} & Application Note: Grundlagen BLDC Motoren. \\
\hline
Elektrotechnik - Antriebe & Stepping Motors Fundamentals & 10 & \href{https://www.microchip.com/en-us/application-notes/an907}{Link} & Application Note: Grundlagen Schrittmotoren. \\
\hline
Elektrotechnik - Antriebe & Stepping Motors Fundamentals & 7 & \href{https://download.beckhoff.com/download/document/application\_notes/dk9222-0410-0014.pdf}{Link} & Application Note: Grundlagen Schrittmotoren. \\

Elektrotechnik - Antriebe & Stepper Motor Reference & 7 & \href{https://www.silabs.com/documents/public/application-notes/an155.pdf}{Link} & Application Note: Grundschaltungen Schrittmotoren. \\
\hline
Elektrotechnik - Energiemanagement & Li-Ion Batterie & 5 & \href{https://learning.oreilly.com/library/view/lithium-ion-batteries/9781466557338/}{Link} & Buch: Lithium-Ionen Batterien. \\
\hline
Elektrotechnik - Energiemanagement & Li-Ion Basics & 8 & \href{https://link.springer.com/book/10.1007/978-3-662-53071-9}{Link} & Buch: Batterietypen. \\
\hline
Elektrotechnik - Energiemanagement & Recycling Li-Ion; Li-Ion & / & \href{https://link.springer.com/book/10.1007/978-3-319-70572-9}{Link} & Buch: Recycling. \\
\hline
Elektrotechnik - Energiemanagement & Li-Ion Battery & 10 & \href{https://learning.oreilly.com/library/view/fundamentals-and-applications/9781118414781/}{Link} & Buch: Verschiedene Batterietypen. \\
\hline
Elektrotechnik - Energiemanagement & NiCad Battery Charge & 5 & \href{https://www.servocity.com/nicad-vs-nimh-batteries}{Link} & Beschreibung: NiCad vs. NiMH Batterien. \\
\hline
Elektrotechnik - Energiemanagement & NiCad Battery Basics & 10 & \href{https://learning.oreilly.com/library/view/rechargeable-batteries-applications/9780750670067/}{Link} & Buch: Grundlagen Nickel-Batterien, Ladevorgänge. \\
\hline
Elektrotechnik - Energiemanagement & Lead Acid Batteries; Batteries; Ni-Cd Batteries & 10 & \href{https://app.knovel.com/kn/resources/kpRBHPA003/toc}{Link} & Buch: Verschiedene Batterietypen sowie Ladeverfahren. \\
\hline
Elektrotechnik - Energiemanagement & Lead Acid Battery & 6 & \href{https://www.researchgate.net/publication/357913548\_LEAD-ACID\_BATTERY}{Link} & Research Paper über Blei-Akkumulatoren. \\
\hline
Elektrotechnik - Energiemanagement & Lead Acid Battery Charge & 4 & \href{https://www.ti.com/lit/an/slua055/slua055.pdf}{Link} & Application Note über Ladeverfahren zu Blei-Akkumulatoren. \\
\hline
Elektrotechnik - Energiemanagement & Lead Acid Battery & 2 & \href{https://ieeexplore.ieee.org/document/8684661}{Link} & Research Paper zu Blei-Akkumulatoren. \\

Elektrotechnik - Energiemanagement & Battery Management; Li-Ion Battery & 8 & \href{https://learning.oreilly.com/library/view/battery-management-system/9781119154006/}{Link} & Buch über Batteriemanagementsysteme für Li-Ion Akkus. \\
\hline
Elektrotechnik - Energiemanagement & Battery Management; Li-Ion Battery & 7 & \href{https://app.knovel.com/kn/resources/kpSALIBM02/toc}{Link} & Buch über Batteriemanagement und Li-Ion Akkus. \\
\hline
Elektrotechnik - Energiemanagement & Battery Management; Li-Ion Battery & 6 & \href{https://app.knovel.com/kn/resources/kpBMSLLIB4/toc}{Link} & Buch über Batteriemanagement und Li-Ion Akkus. \\
\hline
Elektrotechnik - Sensoren & LiDAR und Ultraschall & 3 & \href{https://www.carwow.de/automagazin/auto-lexikon/auto-technologie/lidar-sensor-auto}{Link} & Unterschied von LiDAR und Radar für Abstandsmessung. \\
\hline
Elektrotechnik - Sensoren & Abstandsmessung & 4 & \href{https://ch.farnell.com/dfrobot/sen0259/laser-entfernungssensor-lidar/dp/3769959?cfm=true}{Link} & Möglicher LiDAR Sensor mit Time-of-Flight. \\
\hline
Elektrotechnik - Sensoren & Abstandsmessung & 4 & \href{https://www.az-delivery.de/products/3er-set-hc-sr04-ultraschallmodule}{Link} & Möglicher Ultraschallsensor. \\
\hline
Elektrotechnik - Sensoren & Pfadfindung & 5 & \href{https://intorobotics.com/types-of-sensors-for-target-detection-and-tracking/}{Link} & Verschiedene Sensoren für die Pfadfindung. \\
\hline
Elektrotechnik - Sensoren & Pfadfindung & 4 & \href{https://www.az-delivery.de/products/linienfolger-modul-mit-tcrt5000-und-analog-ausgang}{Link} & Möglicher Infrarotsensor für die Pfadfindung. \\
\hline
Elektrotechnik - Sensoren & Pfadfindung & 3 & \href{https://kiranjoy.blog/2018/08/19/calculate-speed-using-hall-effect-sensor/}{Link} & Geschwindigkeit und Strecke berechnen mit Hallsensor. \\
\hline
Elektrotechnik - Sensoren & Streckenerkennung & 8 & & \\
\hline
Maschinenbau & Mecanum Wheels Overview & 5 & \href{https://intapi.sciendo.com/pdf/10.2478/kbo-2022-0086}{Link} & Überblick über Mecanumräder und deren Verwendungszweck in der Industrie. \\
\hline
Maschinenbau & Räder & 7 & \href{https://www.quasi.ai/choosing-the-right-robotic-wheels-for-your-amr/}{Link} & Überblick und Auswahl verschiedener Rädertypen für einen Roboter. \\
\hline
Maschinenbau & Greifer & 7 & \href{https://qviro.com/blog/introduction-grippers/}{Link} & Funktionsweise von verschiedenen Greifermechanismen. \\
\hline
Maschinenbau & Greifer & 4 & \href{https://www.freise-automation.de/produkte/greifer-handhabungstechnik/}{Link} & Auswahl an Greifern und Linearführungen. \\

Maschinenbau & Greifer & 6 & \href{https://automationspraxis.industrie.de/handling/greifer-fuer-roboter-grundlagen-funktion-und-hersteller/}{Link} & Funktionsweise von verschiedenen Greifermechanismen. \\
\hline
Maschinenbau & Linearführung & 5 & \href{https://blog.item24.com/automatisierte-produktion/mit-lineartechnik-automation-einfach-umsetzen/}{Link} & Überblick an Linearführungen. \\
\hline
Maschinenbau & Material & 2 & \href{https://www.evsint.com/de/materials-to-build-a-robot/}{Link} & Materialauswahl für Chassis. \\
\hline
Maschinenbau & Roboterkinematik & 7 & \href{https://www-home.htwg-konstanz.de/\~{}bittel/msi\_robo/Vorlesung/02\_Roboterkinematik.pdf}{Link} & Roboterkinematik für fahrende Systeme inklusive Linienverfolgung. \\
\hline
Maschinenbau & Bewegungsarten & 5 & \href{https://www.scinexx.de/dossierartikel/rollen-oder-laufen/}{Link} & Verschiedene Bewegungsarten für Roboter. \\
\hline
Maschinenbau & Robotik & 6 & \href{https://www.hs-koblenz.de/fileadmin/media/profiles/ingenieurwesen\_elektrotechnik\_und\_informationstechnik/ross/ROB/Material/handout.pdf}{Link} & Grundlagen der Robotik. \\
\end{longtable}

\end{document}
