\documentclass[main.tex]{subfiles} % Subfile-Class

% ============================================================================== %
%                            Subfile document                                    %
% ============================================================================== %

\begin{document}

% Template

\subsection{Aufgabenstellung}

Die aktuelle Aufgabenstellung des Moduls Produktentwicklung PREN 1 der Hochschule Luzern stellt uns als interdisziplinäres Team vor die Herausforderung, ein autonomes Fahrzeug zu entwickeln, das sich in einem vorgegebenen Wegenetzwerk optimal navigieren kann. Ziel ist es, das Fahrzeug so zu konzipieren, dass es Hindernisse erkennt, gesperrte Bereiche meidet und unter Berücksichtigung der vorab unbekannten Einschränkungen die kürzeste Route vom Start- zum Zielpunkt findet.

Unser Team setzt sich aus Studierenden der Studiengänge Elektrotechnik, Informatik, Maschinenbau und Digital Engineering zusammen, was uns die Möglichkeit bietet, verschiedene technische und methodische Kompetenzen zu vereinen. In der Projektphase von PREN 1 entwickeln wir ein Gesamtkonzept, das auf einem morphologischen Kasten basiert und verschiedene Lösungsvarianten systematisch vergleicht. Ziel ist es, die technische Machbarkeit zu bewerten und erste funktionale Prototypen zu erstellen, die als Validierungsbasis für das weiterführende Modul PREN 2 dienen.

Die Aufgabe erfordert die Entwicklung eines autonomen Systems, das auf einem Netzwerk aus Leitlinien operiert. Zu den wichtigsten Anforderungen gehören:
\begin{itemize}
    \item \textbf{Erkennung gesperrter Wegpunkte:} Diese sind durch Pylonen gekennzeichnet und müssen vom Fahrzeug selbstständig detektiert werden.
    \item \textbf{Bewältigung von Hindernissen:} Das Fahrzeug soll Hindernisse aktiv von der Strecke entfernen und diese an die ursprüngliche Position zurückstellen.
    \item \textbf{Anpassung an veränderte Bedingungen:} Nicht vorhandene Streckenabschnitte müssen vom System zuverlässig als nicht passierbar erkannt werden.
    \item \textbf{Autonome Zielfindung:} Über eine Zielauswahl vor dem Start (Positionen A, B oder C) soll das Fahrzeug den kürzesten und effizientesten Weg zum Zielpunkt finden.
\end{itemize}

Zudem werden bei der Umsetzung strenge Vorgaben zu Dimensionen, Gewicht und Autonomie des Systems berücksichtigt. Dies umfasst die Integration sämtlicher Hardware-Komponenten in das Fahrzeug sowie die Sicherstellung eines störungsfreien und sicheren Betriebs. Weiterhin erarbeiten wir in PREN 1 erste Simulationen, die das Verhalten des Systems unter realitätsnahen Bedingungen analysieren und optimieren.

Die zentrale wissenschaftliche Herausforderung für unser Team besteht darin, Sensorik, Elektronik und Algorithmen so zu entwickeln und zu kombinieren, dass eine zuverlässige Wegfindung und Hindernisbewältigung gewährleistet wird. Dabei legen wir besonderen Wert auf eine methodische Herangehensweise, die neben technischer Präzision auch Nachhaltigkeitsaspekte und Ressourceneffizienz in den Entwicklungsprozess integriert.

Diese Aufgabe bietet uns die Möglichkeit, theoretisches Wissen praktisch anzuwenden und gleichzeitig Kompetenzen in interdisziplinärer Zusammenarbeit und systematischer Produktentwicklung zu vertiefen.

\end{document}