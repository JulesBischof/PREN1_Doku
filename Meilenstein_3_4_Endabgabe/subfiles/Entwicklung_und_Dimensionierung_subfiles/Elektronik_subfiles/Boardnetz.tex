\documentclass[main.tex]{subfiles} % Subfile-Class


% ============================================================================== %
%                            Subfile document                                    %
% ============================================================================== %

\begin{document}

% Template

\subsubsection{Boardnetz}

Der folgende Abschnitt beschäftigt sich mit der Energieversorgung des
Pfadfinders. Wie die Evaluation der Energieversorgung ergeben hat, siehe Anhang
XXXXXXXXXX, wird der Roboter über einen 4S-LiPo mit Energie versorgt. Die
Ausgangsspannung von 14.4V wird auf dem selbst entwickelten DC-PowerBoard
mittels eines Abwärtsstellers auf 12V geregelt. Dieses Board bietet ausserdem
die Möglichkeit, über ein Netzteil in einem Bereich von 12 bis 18 V gespiesen
zu werden und dabei den Akku selbsttätig zu trennen. Dadurch ist der Roboter
bei Entwicklungs- und Einrichtungsaufgaben nicht zwingend auf die
Energieversorgung via Batterie angewiesen, sondern kann auch über ein Netzteil
betrieben werden.

Die Batteriespannung und der Batteriestrom werden über einen ADC überwacht und
können via $I^2C$ vom Raspberry Pi ausgelesen werden.

Die Batterieschutzbeschaltung überwacht die Zellspannungen des LiPo-Akkus
jeweils getrennt und trennt die Batterie vom Gerät, falls eine der Zellen in
den Unterspannungsbereich gerät. Dieser Zustand wird über eine rot leuchtende
LED signalisiert. Die Zell-Balancierung ist nicht auf dem Pfadfinder selbst
umgesetzt. Für diesen Zweck ist ein Ladegerät desselben Herstellers gekauft -
welches die Zellen aktiv ausbalancieren kann.

Das Schema dieses PCB's, sowie sein Layout sind im Anhang beigefügt.
Abbildung~\ref{PowerBoard_Ansicht} zeigt eine 3D-Ansicht dieses PCB's.

\begin{figure}[H]
    \centering
    \includegraphics[width = 0.75\linewidth]{fig_Boardnetz/PowerDistributionBoard.jpg}
    \caption{Ansicht PowerBoard}~\label{PowerBoard_Ansicht}
\end{figure}

\paragraph{Boardnetz - Spannungsverteilung}
Jeder PCB wird über 12V mit Spannung versorgt. Auf den entsprechenden PCB's
werden die geregelten 12V zuerst über einen DC-DC Abwärtssteller auf 6V
heruntergebracht und dann über einen LDO auf 5V bzw. 3V3 aktiv gefiltert.
Dadurch wird verhindert, dass sich Spannungsspitzen oder Ripples durch zum
Beispiel die anlaufenden Motoren im gesamten System verteilen und andere
Elektronik stört. Die jeweils vorgeschalteten Gleichtaktdrosseln verhindern,
dass sich Störungen, verursacht durch laufende Motoren und deren EMI, im System
verteilen. Das Blockschaltbild in Abbildung XXXXXXXXX zeigt die
Spannungsverteilung konzeptionell. %%%%%% ABBILDUNG BLOCKSCHALTBILD

\end{document}
