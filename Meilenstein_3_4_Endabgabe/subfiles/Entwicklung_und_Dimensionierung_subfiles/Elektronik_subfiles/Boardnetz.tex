\documentclass[main.tex]{subfiles} % Subfile-Class


% ============================================================================== %
%                            Subfile document                                    %
% ============================================================================== %

\begin{document}

% Template

\subsubsection{Boardnetz}

Der folgende Abschnitt beschäftigt sich mit der Energieversorgung des
Pfadfinders. Wie die Evaluation der Energieversorgung ergeben hat, siehe Anhang
XXXXXXXXXX, wird der Roboter über einen 4S-LiPo mit Energie versorgt. Die
Ausgangsspannung von 14.4V wird auf dem selbst entwickelten DC-PowerBoard
mittels eines Abwärtsstellers auf 12V geregelt. Dieses Board bietet ausserdem
die Möglichkeit, über ein Netzteil in einem Bereich von 12 bis 18 V gespiesen
zu werden. Dadurch ist der Roboter bei Entwicklungs- und Einrichtungsaufgaben
nicht zwingend auf die Energieversorgung via Batterie angewiesen.

Die Batteriespannung und der Batteriestrom wird über einen ADC überwacht und
kann via $I^2C$ vom Raspberry Pi ausgelesen werden.

Die Batterieschutzbeschaltung überwacht die Zellspannungen des LiPo-Akkus
jeweils separat und trennt die Batterie vom Gerät, falls eine der Zellen in den
Unterspannungsbereich gerät. Dieser Zustand wird über eine rot leuchtende LED
signalisiert. Die Zell-Balancierung ist nicht auf dem Pfadfinder selbst
umgesetzt. Für diesen Zweck ist ein Ladegerät desselben Herstellers gekauft -
welches die Zellen aktiv ausbalanciert.

Jeder PCB wird über 12V mit Spannung versorgt. Auf dem entsprechenden PCB
werden die 12V zuerst über einen DC-DC Abwärtssteller auf 6V heruntergeregelt,
und dann über einen LDO noch auf 5V bzw. 3V3 aktiv gefiltert. Dadurch wird
verhindert, dass sich Spannungsspitzen oder Ripples durch die anlaufenden
Motoren im gesamten System verteilen und die andere Elektronik stört. Die
jeweils vorgeschalteten Gleichtaktdrosseln verhindern, dass sich Störungen,
verursacht durch laufende Motoren und deren EMI, im System verteilen.

Das Schema dieser Schaltung ist Anhang XXXXXXXXXXXXX zu entnehmen, sowie als
Blockschaltbild in Abbildung XXXXXXXXXXX gezeigt.

\end{document}
