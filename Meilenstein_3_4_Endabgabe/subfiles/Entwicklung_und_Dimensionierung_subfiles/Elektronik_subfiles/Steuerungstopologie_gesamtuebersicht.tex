\documentclass[main.tex]{subfiles} % Subfile-Class


% ============================================================================== %
%                            Subfile document                                    %
% ============================================================================== %

\begin{document}

% Template

\subsubsection{Steuerungstopologie und Gesamtübersicht}

Anhang~\ref{appendix:Blockschaltbild_Elektrokonzept} zeigt die Gesamtübersicht
der Steuerung zum Zeitpunkt von PREN1. Ziel ist es, im Sinne der
Gewaltenteilung jede Steuerungseinheit für sich abgeschlossen zu betrachten. So
kann sowohl die Zuständigkeit der beiden Elektronikentwickler gut aufgeteilt
werden, als auch Schnittstellen zwischen Funktionseinheiten sehr eindeutig
benannt werden.

\subsubsection*{Die verschiedenen Funktionsgruppen und ihre PCBs}
Es werden total die 4 folgenden Funktionsgruppen unterschieden:

\begin{description}
      \item[Raspberry HAT] Dieser PCB hat die Aufgabe, die nötige Peripherie für den
            Raspberry Pi zur Verfügung zu stellen. Dies beinhaltet die Anbindung
            verschiedener Taster an die entsprechenden GPIOs sowie eine
            Spannungsversorgung. Der Raspberry Pi soll ausserdem über einen Summer Signale
            wie das Erreichen des Zielpunktes signalisieren können. Weiter wird sein
            UART-Kanal für die Kommunikation aufbereitet.
      \item[Motion Controller] Dieser PCB bietet die Schnittstelle aller Steuereinheiten zu
            den Antrieben. Via UART werden diesem PCB Steuerkommandos wie \textit{fahren
                  mit Geschwindigkeit xxx}, \textit{Fahrzeug um xxx° drehen}, \textit{Hindernis
                  erkannt: Anhalten und warten} und viele weitere mitgeteilt. Er selbst
            beinhaltet Hardware zum Auszählen der Encoder sowie des Gyroskops und noch
            einige digitale Ein-/Ausgänge und Reserve-Sensorschnittstellen.
      \item[Grip Controler] Dieser PCB steuert, wie der Name erahnen lässt, alles rund um
            das Greifen und um den Umgang mit Hindernissen. Er erkennt das Hindernis und
            kann dem Motion Controller via UART Kommandos erteilen, wie das Anhalten oder
            dass das Fahrzeug gedreht werden soll.
      \item[Power Board] Das Power Board bietet die Spannungsaufbereitung für den Roboter.
            Hier werden aus den 14.4V Batteriespannung 12V Boardnetzspannung eingeregelt.
            Weiter wird die Batterie überwacht und bei drohender Unterspannung getrennt.
            Ausbalanciert wird eben diese Batterie über das externe Ladegerät. Wird ein
            14...18V Netzteil an diesem Board angeschlossen, wird die Batterie vom Fahrzeug
            getrennt, was Einricht- und Entwicklungsarbeiten erlaubt, ohne dass dazu eine
            Batterie benötigt wird.
\end{description}

Zum Stand von PREN1 wäre der Motion Controller grundsätzlich in der Lage, die
Aufgabe des Greifers ebenfalls zu übernehmen. Wenn sich der noch nicht
abgeschlossen getestete Greifmechanismus nicht eignet und mehr Motoren oder
Sensoren benötigt, hätte es auf diesem nicht mehr genug Schnittstellen für
Sensorik und Aktorik zur Verfügung. Daher wird mit dem separaten Grip
Controller geplant.

\subsubsection*{Kommunikationsschnittstellen}
Da laufende Motoren, Schaltregler und andere Leistungselektronik auf engem Raum mit
Kommunikations- und anderer feinfühliger Elektronik verbaut werden müssen, wird die UART-Kommunikation
über eine RS422-Schnittstelle differentiell übertragen. Die Kommandos und das Protokoll werden erst
im Nachfolgemodul beim Inbetriebnehmen dieser PCBs definiert. Angedacht ist eine Kommunikation
über eine Kommandolänge von beispielsweise 32- oder 64-Bit, worin eine Adresse definiert ist, ein Kommando und
darüber hinaus noch genügend Platz für zu übertragende Parameter. Mögliche Parameter können dabei
eine Fahrzeuggeschwindigkeit, ein Drehwinkel oder etwa Errorcodes, welche zurück zum Raspberry Pi übertragen werden, sein.

Die Adressenvergabe ist geplant, da der Raspberry Pi auf Basis dieses
Blockschaltbilds grundsätzlich nicht in der Lage ist, auf direktem Wege mit dem
Greifcontroller zu kommunizieren. Dies muss er grundsätzlich auch nicht. Um
allerdings trotzdem eben diesen Fall abzudecken, sollen Adressen von $0 \dots
      2$ (2-Bit) vergeben werden. So kann eine einfache Punkt-zu-Punkt-Übertragung
ausreichend sein, um die drei Steuerungen miteinander kommunizieren zu lassen.

\subsubsection*{Software}
% ToDo
!!! (ABBILDUNG ERSTELLEN GEPLANTE TASKS IN FREERTOS / EMBEDDED SOFTWARE) !!!

\subsubsection*{Boardnetz}
Auf dieses wird in der Sektion~\label{sec:Boardnetz} im Detail eingegangen. Kurz ausgedrückt
besitzt jedes Board eine eigene Spannungswandlung von 12V auf 5V respektive 3.3V, welche vom
Power Board zur Verfügung gestellt werden.

\end{document}
