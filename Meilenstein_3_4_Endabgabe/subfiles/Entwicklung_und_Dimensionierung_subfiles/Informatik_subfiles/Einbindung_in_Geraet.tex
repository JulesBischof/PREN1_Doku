\documentclass[main.tex]{subfiles} % Subfile-Class

\begin{document}

\subsubsection{Einbindung in Gerät}
Benötigte Schnittstellen, Sensorik \& Co. Ausserdem: Wo stehen noch Risiken?
Thema Messtoleranzen \& Co. ?\\

Der beschriebene Algorithmus zur Navigation und Exploration eines
unbekannten Graphen benötigt spezifische Schnittstellen und Sensoren,
um seine Funktionalität sicherzustellen. Diese Elemente bilden die
Grundlage für die präzise Erfassung der Umgebung und die
Entscheidungsfindung des Roboters.

\paragraph{Benötigte Schnittstellen und Sensorik}
Um den Algorithmus korrekt auszuführen, werden folgende Sensoren und
Schnittstellen benötigt:

\begin{itemize}
  \item \textbf{Gyroskop}:
    Das Gyroskop liefert Feedback zur Ausrichtung des Roboters und
    ermöglicht es, die Bewegungsrichtung präzise zu bestimmen. Dies
    ist essenziell, um die interne Karte zu zeichnen und die
    Orientierung innerhalb des Graphen zu bewahren.

  \item \textbf{Motor-Feedback}:
    Die Informationen aus den Motoren, wie die zurückgelegte Distanz,
    sind entscheidend, um die Länge der Kanten im Graphen zu kennen.
    Dadurch kann der Algorithmus die relative Position von Knoten und
    Kanten in der internen Karte bestimmen.

  \item \textbf{Lidar}:
    Der Lidar-Scanner dient der frühzeitigen Erkennung von
    Hindernissen wie Pylonen. Diese Informationen werden verwendet,
    um Kanten im Graphen als unpassierbar zu markieren, bevor der
    Roboter diese streift. Dies erhöht die Effizienz und Sicherheit
    bei der Navigation.

\end{itemize}

Durch die Kombination dieser Sensoren kann der Roboter eine dynamisch
aktualisierte Karte erstellen, die sowohl besuchte als auch
unbesuchte Knoten und Kanten enthält.

\paragraph{Umgang mit Schranken}
Der Algorithmus behandelt Strecken mit Schranken vorerst bewusst wie
reguläre Strecken, ohne sie bei der Pfadauswahl stärker zu gewichten.
Dies reduziert die Komplexität bei der Routenplanung. Je nach
Geschwindigkeit, mit der Hindernisse umplatziert werden können,
könnte dieser Ansatz jedoch reevaluiert werden, um Schranken in
Zukunft in die Gewichtung einzubeziehen.

\paragraph{Risiken und Messtoleranzen}
Trotz der robusten Integration von Sensoren gibt es spezifische
Risiken und Herausforderungen, die berücksichtigt werden müssen:

\begin{itemize}
  \item \textbf{Messtoleranzen}:
    Abweichungen bei den Sensorwerten können zu Fehlern in der
    internen Karte führen. Beispielsweise können:
    \begin{itemize}
      \item \textbf{Schlupf an den Rädern} die zurückgelegte Distanz
        falsch messen (Motor-Feedback).
      \item \textbf{Drift im Gyroskop} zu Abweichungen in der
        Orientierung führen.
    \end{itemize}
    \textbf{Lösung}: Im Algorithmus nicht mit exakten Werten, sondern
    Wertebereichen rechnen. Dies erlaubt Fehler in den Messungen.

  \item \textbf{Fehlinterpretation von Pylonen}:
    Temporäre Hindernisse, wie ein Bein eines Zuschauers, könnten vom
    Lidar fälschlicherweise als Pylonen erkannt werden. Dies würde
    dazu führen, dass Strecken oder Knoten unnötig als gesperrt
    markiert werden.\\
    \textbf{Lösung}: Zuschauer bitten Abstand zu nehmen und eine
    Verifizierung, bei der Hindernisse mehrfach von verschiedenen
    Seiten erkannt werden müssen, bevor sie dauerhaft in die Karte
    aufgenommen werden.

  \item \textbf{Sonderfall: Gesperrter Knoten hinter einem freien Knoten}:
    Wenn sich ein gesperrter Knoten direkt hinter einem freien Knoten
    befindet (innerhalb von 2 Metern), könnte der Algorithmus
    fälschlicherweise auch den freien Knoten als gesperrt markieren.\\
    \textbf{Lösung}: Der Roboter kann den Knoten mit vermutetem Pylon
    aus einem anderen Winkel reevaluieren, um eine fehlerhafte
    Markierung zu vermeiden. Dies minimiert das Risiko und erhöht die
    Genauigkeit der internen Karte.

\end{itemize}

\paragraph{Fazit}
Die Kombination der Sensoren Gyroskop, Motor-Feedback und Lidar
bildet die Grundlage für die präzise Navigation des Roboters. Trotz
einiger Risiken, wie Messtoleranzen und Fehlinterpretationen, bietet
der Algorithmus durch Reevaluation und rechnen mit Wertebereichen
eine robuste Lösung, um ungenaue Messungen abzufangen. Diese
Massnahmen reduzieren die genannten Risiken und erlauben eine
zuverlässigere Navigation.

\end{document}
