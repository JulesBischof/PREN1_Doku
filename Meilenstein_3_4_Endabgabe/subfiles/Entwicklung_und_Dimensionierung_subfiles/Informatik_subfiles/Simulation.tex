\documentclass[main.tex]{subfiles} % Subfile-Class

\begin{document}

\subsubsection{Simulation}

Die Simulation dient als Werkzeug zur Validierung und Optimierung der 
Navigationsalgorithmen in einer virtuellen Umgebung. 
Sie ermöglicht es, das Verhalten des Roboters unter verschiedenen 
Szenarien zu testen, bevor physische Prototypen entwickelt werden. 
Der implementierte Simulator unterstützt Funktionen wie die Simulation 
von Hindernissen, gesperrten Wegpunkten und alternativen Routen.

Der vollständige Sourcecode des Simulators ist im zugehörigen 
GitHub-Repository abrufbar:
\href{https://github.com/tramasys/simpren}{https://github.com/tramasys/simpren}

\paragraph{Konzeption}

Der Simulator wurde so konzipiert, dass er eine flexible und
intuitive Plattform zur Analyse und Optimierung von
Navigationsalgorithmen bereitstellt. Die Architektur basiert auf vier
Hauptkomponenten, die über gemeinsam genutzte Datenstrukturen und
Stores miteinander kommunizieren. Das folgende Diagramm verdeutlicht
den Aufbau des Simulators und die Interaktion zwischen den Komponenten:

\begin{figure}[H]
  \centering
  \includegraphics[width=0.75\textwidth]{./fig_Simulation/SimulatorKomponentenDiagramm.png}
  \caption{Aufbau des Simulators und dessen
  Komponenten}~\label{fig:SimulatorKomponentenDiagramm}
\end{figure}

Die zentralen Komponenten sind:
\begin{itemize}
  \item \textbf{GraphViewer:} Visualisiert den Graphen und ermöglicht
    die Aktualisierung der Knoten und Kanten im
    \textit{GraphStructure}-Store. Nutzer:innen können über diese
    Komponente den aktuellen Zustand des Graphen einsehen und
    manuelle Änderungen vornehmen.
  \item \textbf{DashboardViewer:} Bietet eine Benutzeroberfläche zur
    Konfiguration der Simulation und interagiert mit dem
    \textit{Configs}-Store, um Simulationsparameter zu lesen oder zu
    schreiben. Hier können Betriebsmodi ausgewählt und
    Simulationseinstellungen angepasst werden.
  \item \textbf{GraphExplorer:} Die zentrale Komponente des
    Simulators, die den Graphen analysiert, Pfade berechnet und Logs
    im \textit{Logs}-Store speichert. Der \textit{GraphExplorer}
    übernimmt die Hauptlogik der Simulation und entscheidet über die
    Navigationsstrategien.
  \item \textbf{LogViewer:} Ermöglicht die Ansicht und Analyse der im
    \textit{Logs}-Store gespeicherten Simulationslogs. Nutzer:innen
    können die Entscheidungen und den Fortschritt des Algorithmus
    nachvollziehen.
\end{itemize}

Die Kommunikation zwischen den Komponenten erfolgt über drei zentrale Stores:
\begin{itemize}
  \item \textbf{GraphStructure:} Speichert den Zustand des Graphen,
    inklusive der Knoten und Kanten, und wird von
    \textit{GraphViewer} und \textit{GraphExplorer} genutzt.
  \item \textbf{Logs:} Enthält alle Simulationslogs, die von
    \textit{GraphExplorer} geschrieben und von \textit{LogViewer}
    gelesen werden können. Diese Logs dokumentieren die Schritte und
    Entscheidungen während der Simulation.
  \item \textbf{Configs:} Speichert die Simulationsparameter, die von
    \textit{DashboardViewer} konfiguriert und von
    \textit{GraphExplorer} gelesen werden. Dazu gehören Einstellungen
    wie Betriebsmodi, Parameter für Algorithmen und
    benutzerdefinierte Anpassungen.
\end{itemize}

\paragraph{Allgemeine Funktionen}

Der entwickelte Simulator dient der Visualisierung, Analyse und
Optimierung von Algorithmen zur Navigation eines Roboters in einem
unbekannten Graphennetzwerk. Er wurde mit dem Ziel konzipiert, eine
flexible und intuitive Plattform für die Simulation unterschiedlicher
Szenarien zu bieten. Die Funktionen sind so ausgelegt, dass sowohl
die interaktive Bedienung als auch automatisierte Simulationen und
visuelle Darstellungen unterstützt werden.

Es wurden verschiedene Algorithmen im Simulator implementiert, wie
Dijkstra, A* und D*Lite. Letztendlich hat sich das Informatikteam
dafür entschieden, einen eigenen Algorithmus zu entwickeln. Im Laufe
der Entwicklung wurde immer mehr klar, dass es durch das Wegnehmen
von Strecken auf dem Graph zu einem Navigationsproblem kommen kann.
Deswegen hat man sich dafür entschieden, nicht mit den Algorithmen
Dijkstra und D*Lite weiterzufahren. Vollständigkeitshalber sind sie
aber noch im Simulator enthalten.

\paragraph{Visualisierung und Steuerung}

Die Steuerung und Visualisierung des Graphen erfolgt über ein
benutzerfreundliches Dashboard. Es stehen drei wesentliche Optionen
zur Verfügung~\ref{fig:DashboardGraph}:

\begin{itemize}
  \item \textbf{Randomized Graph}:
    Generiert einen neuen, zufällig aufgebauten Graphen, der als
    Grundlage für die Simulation dient.

  \item \textbf{Reset Graph}:
    Setzt den Graphen in seinen Ursprungszustand zurück, entfernt
    alle Barrieren, Pylonen und andere Modifikationen.

  \item \textbf{Reset State}:
    Stellt den Zustand aller Knoten und Kanten auf ihre
    Standardeinstellungen zurück. Unterschiedliche Stati wie
    \emph{probed} (gelb), \emph{visited} (grün), \emph{restricted}
    (rot) oder \emph{finished} (blau) werden visuell hervorgehoben
    und erlauben eine präzise Nachverfolgung des Algorithmus.
\end{itemize}

\begin{figure}[H]
  \centering
  \fbox{\includegraphics[width=0.75\textwidth]{./fig_Simulation/Graph.png}}
  \caption{Screenshot des Bereichs für den Graph im Simulator (GUI)}~\label{fig:DashboardGraph}
\end{figure}

\paragraph{Manuelle Bearbeitung von Knoten und Kanten}

Der Simulator erlaubt eine intuitive manuelle Bearbeitung von Knoten
und Kanten im Graphen direkt über die Benutzeroberfläche. Dies
ermöglicht es dem Nutzer, die Simulation individuell anzupassen und
spezifische Szenarien zu testen~\ref{fig:DashboardGraph}:

\begin{itemize}
  \item \textbf{Knoten bearbeiten}:
    Durch einfaches Klicken auf einen Knoten wird ein Pylon auf
    diesen gesetzt. Der Pylon repräsentiert ein Hindernis auf dem
    Knoten. Ein erneutes Klicken entfernt den Pylon wieder, wodurch
    der Knoten frei wird.

  \item \textbf{Kanten bearbeiten}:
    Kanten durchlaufen drei Zustände, die durch wiederholtes Klicken
    in einem zyklischen Ablauf geändert werden können:
    \begin{enumerate}
      \item \textit{Normale Strecke}: Die Kante ist ohne Hindernis befahrbar.
      \item \textit{Entfernte Strecke}: Die Kante wird entfernt, was
        im UI durch eine gestrichelte Linie angezeigt wird.
      \item \textit{Strecke mit Schranke}: Die Kante ist befahrbar,
        hat jedoch eine Schranke, die zusätzliche Fahrzeit oder eine
        andere Einschränkung symbolisiert.
    \end{enumerate}
    Nach einem weiteren Klick kehrt die Kante wieder in den
    Ausgangszustand als normale Strecke zurück. Dieser Zyklus wird im
    UI visuell hervorgehoben, um den aktuellen Zustand klar darzustellen.
\end{itemize}

\paragraph{Funktionen und Modus-Optionen}

Der Simulator bietet drei verschiedene Betriebsmodi. Diese werden
durch Tabs voneinander graphisch getrennt~\ref{fig:DashboardConfig}:

\begin{itemize}
  \item \textbf{Interactive run}:
    Dieser Modus erlaubt es Graph Algorithmen, wie Dijkstra, A* und 
    D*Lite interaktiv zu testen, wobei spezifische Parameter durch 
    den Benutzer angepasst werden können.

  \item \textbf{Parameterized run}:
    In diesem Modus können mehrere Durchläufe automatisiert
    durchgeführt werden. Es ist möglich, Parameter wie die Fahrzeit
    auf verschiedenen Strecken (normale Strecken vs. Strecken mit
    Barrieren), die Anzahl der zu simulierenden Durchläufe (z. B. 100
    Runs) und andere Variablen zu spezifizieren. Der Simulator
    generiert dabei zufällige Graphen und löst diese mittels eines
    heuristikbasierten Tiefensuchalgorithmus (DFS).

  \item \textbf{Explore Map}:
    In diesem Modus wird der heuristikbasierte DFS-Algorithmus
    visuell dargestellt. Der Nutzer kann das Ziel pro Durchlauf
    selbst bestimmen, wodurch der Graph Schritt für Schritt
    aufgedeckt wird. Diese Visualisierung identifiziert auch Knoten,
    die durch mathematische Berechnungen erkannt werden, obwohl der
    Roboter diese nicht direkt besucht hat. Wenn eine neu entdeckte
    Kante eine bereits bekannte Kante schneidet, wird anhand linearer
    Algebra auf die Existenz eines Knotens geschlossen.
\end{itemize}

\begin{figure}[H]
  \centering
  \fbox{\includegraphics[width=0.75\textwidth]{./fig_Simulation/SimulatorConfig.png}}
  \caption{Screenshot des Bereichs für Betriebsmodi und
  Konfiguration im Simulator}~\label{fig:DashboardConfig}
\end{figure}

\paragraph{Log-Analyse}

Der Simulator enthält ein integriertes Log-System, das die
Entscheidungen des Algorithmus dokumentiert. Die Logs können direkt
im Dashboard gelesen oder als CSV-Datei exportiert werden. Folgende
Funktionen stehen zur Verfügung~\ref{fig:DashboardLogs}:

\begin{itemize}
  \item \textbf{Clear Logs}:
    Setzt die Logs vollständig zurück.

  \item \textbf{Export Logs}:
    Exportiert die Logs als CSV-Datei.

  \item \textbf{Clear Screen}:
    Entfernt die Logs aus der Ansicht, ohne diese zu löschen.
\end{itemize}

\begin{figure}[H]
  \centering
  \fbox{\includegraphics[width=0.75\textwidth]{./fig_Simulation/SimulatorLogs.png}}
  \caption{Screenshot des Bereichs für Logs im Simulator}~\label{fig:DashboardLogs}
\end{figure}

\paragraph{Fazit}

Der Simulator bietet eine vielseitige Plattform für die Simulation
und Analyse von Navigationsalgorithmen. Durch die Kombination aus
visueller Darstellung, flexiblen Betriebsmodi und präzisem Logging
ermöglicht er ein umfassendes Verständnis der Funktionsweise und
Leistungsfähigkeit der eingesetzten Algorithmen.

\end{document}
