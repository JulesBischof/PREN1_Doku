\documentclass[main.tex]{subfiles} % Subfile-Class


% ============================================================================== %
%                            Subfile document                                    %
% ============================================================================== %

\begin{document}

% Template

\subsubsection{Chassis und Fahrwerk}

Dieser Abschnitt behandelt die Evaluierung des Fahrwerks in Bezug auf die Radauswahl
und die Chassis-Konstruktion. Dabei werden verschiedene Konzeptlösungen diskutiert und
getestet. Abschliessend wird die finale Entscheidung auf Basis der Testergebnissen 
dokumentiert.

% ===================================================================================
\paragraph{Anforderungen}

\textbf{Gewicht:} \newline
Das Gewicht stellt, wie auch bei allen anderen Baugruppen, einen kritischen Faktor dar. 
Daher basiert das Chassis auf dem Konzept einer einzigen Grundplatte, anstatt einen 
Volumenkörper zu entwerfen, der sämtliche Aktoren, Sensoren und Steuerungskomponenten 
umschliesst. Zudem wird angestrebt, die Konstruktion auf zwei Räder zu beschränken. 
Die Drehbewegung wird mittels Skid Steering realisiert. Um die Stabilität zu 
gewährleisten, benötigt der Roboter jedoch einen dritten Auflagepunkt, dessen genaue 
Umsetzung in der Phase PREN 2 durch verschiedene Tests untersucht wird.

\textbf{Geschwindigkeit:} \newline
Da die Motoren bereits ausgewählt sind und ihre Drehzahl technisch begrenzt ist, 
kann die maximale Fahrgeschwindigkeit lediglich über die Radgrösse definiert werden. 
Von einem Getriebe wird bewusst abgesehen, um Gewicht und Kosten zu reduzieren.

\textbf{Kosten:} \newline
Die Anbauteile des Chassis werden mit einem 3D-Drucker gefertigt, da dem Team 
25 Stunden Druckzeit des HSLU-T\&A-Druckers kostenfrei zur Verfügung stehen. 
Die Kosten für die Räder sollen unter 20 CHF bleiben, um ein grösseres Budget 
für kritische Funktionen bereitzustellen.

% ===================================================================================
\paragraph{Konzeptionierung}

Die Grundplatte des Chassis wird aus einer MDF-Platte gefertigt, in die sämtliche 
Ausschnitte und Öffnungen mittels Lasergravur eingebracht werden. Komponenten werden 
modular mit Anbauteilen aus dem 3D-Druck auf der Grundplatte montiert. Diese modulare 
Konstruktion ermöglicht einfache Anpassungen und hilft, das Gewicht zu reduzieren.

Bereits früh in der Projektphase wurde entschieden, dass die Fortbewegung des Pfadfinders 
mit zwei Rädern und einem dritten Auflagepunkt realisiert werden soll. 
Abbildung~\ref{fig:Radkonzept} zeigt eine erste Konzeptskizze des Chassis, einschliesslich 
der Räder und einer Rutschfläche als drittem Auflagepunkt.

\begin{figure}[H]
    \centering
    \includegraphics[width=0.75\textwidth]{Radkonzept.pdf}
    \caption{Chassis Konzept in Siemens NX}~\label{fig:Radkonzept}
\end{figure}

Die Räder dürfen eine maximale Breite von 20 mm nicht überschreiten, um ausreichend 
Platz auf der Grundplatte für Motoren und Motorentreiber bereitzustellen. Der 
Mindestdurchmesser beträgt 80 mm, da andernfalls die Liniensensoren nicht wie 
vorgesehen montiert werden können. Zudem ermöglicht ein grösserer Durchmesser 
von 80 mm eine maximale Geschwindigkeit von 1.676~m/s, was den 
Projektanforderungen entspricht. Die Berechnung erfolgt gemäss 
folgender Formel:

\[ v_{max} = n_{Motor} \cdot d_{Rad} \cdot \pi = 1.676 \, \frac{\text{m}}{\text{s}} \]

Auf Grundlage der genannten Anforderungen wurde auf verschiedenen Webseiten nach 
passenden Rädern gesucht. Die Ergebnisse sind in Tabelle~\ref{tab:Rad_Parameter} 
zusammengefasst.

\begin{table}[h]                                    % h = here
    \centering
    \begin{tabular}{|c|c|c|c|c|c|}                        % c=centered, l=left, r=right
        \hline
        Hersteller  & Herst. Nr.    & Preis     & Durchmesser   & Material      & Gewicht   \\
                    &               &           &               & Lauffläche    &           \\ \hline
        ServoCity   & 595660        & 9.05 CHF  & 80.3 mm       & Silikon       & 19 g      \\ \hline
        DF-Robot    & FIT0500       & 2.62 CHF  & 80 mm         & Silikon       & -         \\ \hline
    \end{tabular}
    \caption{Rad Parameter}
    \label{tab:Rad_Parameter}
\end{table}

% ===================================================================================
\paragraph{Entscheidung und Fazit}
Von den beiden untersuchten Rädern wurde das Modell FIT0500 von DF-Robot ausgewählt. 
Obwohl die Aufnahme des ServoCity-Rades technisch besser geeignet wäre, wurden die 
erheblichen Lieferkosten von 70 CHF als entscheidender Nachteil gewertet.

\end{document}
