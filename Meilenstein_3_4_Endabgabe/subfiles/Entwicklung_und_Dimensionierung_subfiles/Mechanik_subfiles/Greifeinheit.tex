\documentclass[main.tex]{subfiles} % Subfile-Class


% ============================================================================== %
%                            Subfile document                                    %
% ============================================================================== %

\begin{document}

% Template

\subsubsection{Greifeinheit}

Dieser Abschnitt behandelt die Entwicklung eines Greifarmkonzepts, das die Auswahl 
geeigneter Greifelemente und die Konstruktion der Greifereinheit umfasst. 
Dabei werden verschiedene Konzeptlösungen analysiert und getestet.
Abschliessend wird die finale Entscheidung auf Basis der Testergebnisse dokumentiert.
% ===================================================================================
\paragraph{Anforderungen}

\textbf{Greifkraft:} \newline
Die Greifkraft muss ausreichend dimensionert sein, um das Hindernis sicher greifen zu können. 
Dabei ist die Haftreibung und die Anpresskraft zu berücksichtigen.

\textbf{Höhenverstellung:} \newline
Die Höhenverstellung muss an das Gewicht des zu hebenden Hindernisses ausgelegt sein. 
Dafür ist es erforderlich, dass der Motor das erforderliche Drehmoment bereitstellt.

\textbf{Gewicht:} \newline
Das Gewicht des Greiferkonzepts ist so gering wie möglich zu halten, 
um das Gesamtgewicht der Fahrzeugs zu reduzieren. 
Eine Gewichtsreduktion ist ebenfalls bei der bewegten Masse des Greifers von Bedeutung, 
da sie direkte Auswirkungen auf Geschwindigkeit, Agilität und Energieeffizienz hat.

\textbf{Genauigkeit:} \newline
Der Greifer muss den gesamten Bewegungsablauf mit hoher Wiederholgenauigkeit ausführen. 
Dadurch wird sichergestellt, dass das Hindernis zuverlässig innerhalb des vorgegebenen 
Toleranzbereichs positioniert wird.

% ===================================================================================
\paragraph{Konzeptionierung}
\newline
Konzept "Paralellgreifer und Höhenverstellung"
Bild
Mechanismus erklären, etc.
\newline
Konzept "Greifer und Höhenverstellung mit einem Motor"
Bild
Mechanismus erklären, etc.
\newline
Konzept "Gabelstapler"
Bild
Ein Greiferkonzept, das wie ein Gabelstapler funktioniert. Die Gabeln werden in die Öffnungen 
des Hindernis eingeführt und heben dieses anschliessend an. Dieses Design erfordert nur einen 
Motor für die Höhenverstellung. (Jedoch Zielsicherheit beim Treffen der Löcher, Kamera, ka etc.)

% ===================================================================================
\paragraph{Versuche}

Test Paralellgreifer mit Motor

Test Gabelstapler mit Motor

Test Ein Motor Variante (Prototyp)

% ===================================================================================
\paragraph{Entscheidung und Fazit}
Fazit und Entscheid

\end{document}
