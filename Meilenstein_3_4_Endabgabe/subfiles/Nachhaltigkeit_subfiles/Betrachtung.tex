\documentclass[../../main.tex]{subfiles} % Subfile-Class

% ==============================================================================
% %
%                            Subfile document
%            %
% ==============================================================================
% %

\begin{document}

\subsection{Nachhaltigkeitsbetrachtung}

\subsubsection{Ziele der Nachhaltigkeitsbetrachtung}

Das Ziel dieser Nachhaltigkeitsbetrachtung ist es, die Entwicklung
und den Einsatz unseres autonomen Fahrzeugs unter ökologischen und
ökonomischen Gesichtspunkten zu analysieren. Dabei sollen nachhaltige
Praktiken und Materialien identifiziert und integriert werden, um
negative Auswirkungen auf die Umwelt zu minimieren und eine
eventuelle Wiederverwendung der einzelnen Komponenten im privaten
Umfang oder bei nachfolgenden Durchführungen dieses Moduls zu ermöglichen.

\subsubsection{Abgrenzung}

In PREN1 fokussieren wir uns insbesondere auf die
Wiederverwendbarkeit der eingesetzten Komponenten, wobei wir
besonderes Augenmerk auf die Elektronik legen. Wo immer möglich,
verwenden wir High- und Low-Level-Controller wie beispielsweise einen
Raspberry Pi aus bereits vorhandenem Bestand. Wir streben eine
Optimierung des Energieverbrauchs des autonomen Fahrzeugs an, da ein
geringerer Stromverbrauch nicht nur der Umwelt zugutekommt, sondern
auch bei der Erstellung des Fahrzeugs Vorteile bietet. Zudem legen
wir Wert auf kurze Lieferwege für bestellte Komponenten und
bevorzugen Artikel, die wir Nearshore beziehen können.

Aspekte wie die Entsorgung oder das Recycling von nicht
wiederverwendbaren Komponenten am Ende ihrer Lebensdauer
sowie die detaillierte Analyse der gesamten Lieferkette der
verwendeten Materialien werden in dieser Betrachtung nicht
ausführlich behandelt. Ebenso werden soziale Nachhaltigkeitsaspekte
und die Auswirkungen auf globale Nachhaltigkeitsziele ausserhalb des
Projektumfangs nicht vertieft analysiert.

\subsubsection{Bezug zu den Sustainable Development Goals (SDGs)}

Die Entwicklung des autonomen Fahrzeugs steht in Verbindung mit zwei
der von den Vereinten Nationen definierten Sustainable Development Goals:

\begin{itemize}
  \item \textbf{SDG 9 - Industrie, Innovation und Infrastruktur}:
    Durch die Entwicklung eines innovativen autonomen Systems tragen
    wir zur Förderung von nachhaltiger Industrie und Infrastruktur bei.
  \item \textbf{SDG 12 - Nachhaltiger Konsum und Produktion}: Durch
    die Verwendung leichter und effizienter Materialien sowie
    energiesparender Komponenten unterstützen wir nachhaltige
    Produktionsmuster.
\end{itemize}

\end{document}
