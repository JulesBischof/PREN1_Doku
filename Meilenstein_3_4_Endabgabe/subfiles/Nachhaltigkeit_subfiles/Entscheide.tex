\documentclass[../../main.tex]{subfiles} % Subfile-Class

% ==============================================================================
% 
%                            Subfile document
%            
% ==============================================================================

\begin{document}

\subsection{Entscheide}

In diesem Abschnitt werden die Nachhaltigkeitsaspekte bei den
Entscheidungsprozessen in den drei Hauptbereichen des Projekts – Elektronik,
Mechanik und Informatik – dargestellt.

\subsubsection{Elektronik}

\begin{itemize}
      \item \textbf{Komponentenwahl:} Im Rahmen der Komponentenwahl finden Mikrocontroller Verwendung,
            die sich bereits im Besitz einzelner Teammitglieder befinden. Auch die Bauteile und Sensoren für
            den Prototypenbau und Tests befinden sich zum grössten Teil im Besitz der Teammitglieder und
            können somit wiederverwendet werden. Dadurch wird verhindert, dass bei der Herstellung des
            Endproduktes Bauteile bestellt werden, die später keine Anwendung im Endprodukt finden.Des
            Weiteren verfügen die entwickelten Leiterplatten über Reserve-Schnittstellen, wie z. B.
            überzählige digitale Ein- und Ausgänge.Somit müssen bei Änderungen am bestehenden Konzept die Leiterplatten
            nicht neu entwickelt und bestellt werden.Die elektronische Plattform ist modular aufgebaut. Die
            Kommunikationsschnittstelle sowie die Möglichkeit, die einzelnen Controller über eine Adressierung
            individuell anzusprechen, ermöglichen eine einfache und schnelle Systemerweiterung. So kann im Falle nicht ausreichender
            Ein-/Ausgänge zum Beispiel problemlos ein zweiter MotionController, welcher entsprechende Ausgänge zur Verfügung stellt,
            eingesetzt werden. Komponenten von weit entfernten Lieferanten müssen so nur einmal für die
            gesamte Projektlaufzeit bestellt werden. Es wird angestrebt, für sämtliche Leiterplatten, soweit
            dies realisierbar ist, identische Bauteile für Schnittstellen, Steckverbinder und Schalter-ICs zu
            verwenden. Die Verwendung identischer Bauteile für Schnittstellen, Steckverbinder und Schalter-ICs
            ermöglicht die Bereitstellung identischer Ersatzteile für alle Teilsteuerungen und reduziert
            die Komplexität, was die Wiederverwendung der Boards begünstigt.
            Weiter werden vorzugsweise Komponenten verwendet, welche sich in den Lagern des Elektrolabors in der
            Hochschule Luzern befinden - bevor diese neu bezogen werden.

      \item \textbf{Wiederverwendbarkeit:} Der modulare Aufbau der Elektronik ermöglicht die
            Wiederverwendung der Leiterplatten durch andere Teams aus späteren Jahrgängen. Im
            Anschluss an das Projekt erfolgt die Veröffentlichung der entsprechenden Baugruppen
            und Fertigungspläne auf Studenten-Plattformen wie \textit{Studentbox}, um nachfolgenden
            Jahrgängen zur Verfügung zu stehen. Einige Komponenten, wie der Raspberry Pi und
            die Mikrocontroller aus persönlichen Beständen, werden für zukünftige private
            Projekte wiederverwendet.

      \item \textbf{Lieferantenauswahl:} Es wird auf Lieferanten zurückgegriffen, deren Komponenten
            die Hochschule Luzern auch für eigene Entwicklungen verwendet. Dadurch können grössere Bestellungen mit
            anderen Teams und der Hochschule gemeinsam getätigt werden, was Lieferungen in grösseren Mengen erlaubt.
            Zu diesen zählen insbesondere \textit{Mouser}, \textit{DigiKey} und \textit{Conrad}.
            Obwohl die Leiterplatten vom asiatischen Hersteller PCBWAY bezogen werden, wird durch den modularen
            Aufbau der Elektronik die Umweltbelastung so gering wie möglich gehalten.
\end{itemize}

\newpage

\subsubsection{Mechanik}

\begin{itemize}
      \item \textbf{Materialauswahl:}
            Das Chassis besteht aus MDF-Holzplatten und dem 3D-Druck-Material PLA. MDF wird 
            üblicherweise aus Rest- und Abfallfasern hergestellt, was eine ressourcenschonende 
            Nutzung ermöglicht. Die thermische Verwertung von MDF-Platten ist derzeit die gängigste 
            Recyclingmethode, bei der die Platten zur Energiegewinnung genutzt werden. Nach Abschluss 
            des Projekts besteht jedoch die Möglichkeit, diese Materialien in neuen Anwendungen weiter 
            zu verwerten. Ergänzend kommt das 3D-Druck-Material PLA zum Einsatz, das aus nachwachsenden 
            Rohstoffen wie Maisstärke gewonnen wird und unter industriellen Bedingungen biologisch abbaubar 
            ist. Durch das additive Fertigungsverfahren wird Material gezielt nur an den benötigten Stellen 
            aufgetragen, wodurch weniger Abfall im Vergleich zu spanenden Verfahren entsteht. Beide Materialien 
            können lokal beschafft und verarbeitet werden, was den Transportaufwand und damit verbundene 
            Emissionen minimiert.

      \item \textbf{Modulares Design:}
            Ein weiterer zentraler Ansatz zur Förderung der Nachhaltigkeit war das modulare Design des Systems. 
            Die Konstruktion in austauschbaren Baugruppen ermöglicht es, einzelne Komponenten im Falle eines Defekts 
            gezielt zu ersetzen, anstatt das gesamte System erneuern zu müssen. Dieses Vorgehen verlängert die Lebensdauer 
            des Produkts und reduziert den Ressourcenverbrauch erheblich. Am Ende des Projekts können die Baugruppen 
            entweder in neuen Projekten wiederverwendet oder sortenrein recycelt werden. Ein modularer Aufbau hat zudem 
            den Vorteil, dass bei zukünftigen Iterationen nicht alle Bauteile neu gefertigt werden müssen, was den 
            Materialeinsatz weiter reduziert. Trotz des anfänglichen Mehraufwands in der Entwicklungsphase zahlt sich 
            dieses Designprinzip langfristig aus, da es den Bedarf an neuen Ressourcen minimiert und so einen wesentlichen 
            Beitrag zur Ressourcenschonung und Reduktion von Abfall leistet.

\end{itemize}

\subsubsection{Informatik}

\begin{itemize}
      \item \textbf{Verwendung von Open-Source-Software:} Für die
            Entwicklung des Simulators wird ausschliesslich
            Open-Source-Software eingesetzt. Dies fördert die Nachhaltigkeit
            durch Reduzierung von Lizenzkosten und Abhängigkeiten von
            proprietärer Software und unterstützt gleichzeitig die
            Open-Source-Gemeinschaft. Die Nutzung frei verfügbarer Ressourcen
            ermöglicht es, den Quellcode an spezifische Anforderungen
            anzupassen, was zu effizienteren und ressourcenschonenderen
            Anwendungen führt.

      \item \textbf{Nutzung vorhandener Hardware:} Es werden Raspberry Pi
            Einplatinencomputer verwendet, die bereits verfügbar sind. Durch
            die Wiederverwendung bestehender Hardware wird der Bedarf an neu
            produzierten elektronischen Geräten minimiert, was zur
            Reduzierung von Elektroschrott und zur Schonung natürlicher
            Ressourcen beiträgt. Diese Entscheidung ist sowohl ökologisch
            sinnvoll als auch kosteneffizient und unterstützt die
            Nachhaltigkeitsziele des Projekts.
\end{itemize}

\end{document}
