\documentclass[../../main.tex]{subfiles} % Subfile-Class

% ==============================================================================
% %
%                            Subfile document
%            %
% ==============================================================================
% %

\begin{document}

\subsection{Entscheide}

In diesem Abschnitt werden die Nachhaltigkeitsaspekte bei den
Entscheidungsprozessen in den drei Hauptbereichen unseres Projekts –
Elektronik, Mechanik und Informatik – dargestellt.

\subsubsection{Elektronik}

\begin{itemize}
  \item \textbf{Komponentenwahl:} asdf
  \item \textbf{Wiederverwendbarkeit:} asdf
  \item \textbf{Lieferantenwahl:} adsf
\end{itemize}

\subsubsection{Mechanik}

\begin{itemize}
  \item \textbf{Materialauswahl:} asdf
  \item \textbf{Modulares Design:} asdf
  \item \textbf{Energieeffiziente Antriebssysteme:} asdf
\end{itemize}

\subsubsection{Informatik}

\begin{itemize}
  \item \textbf{Verwendung von Open-Source-Software:} Für die
    Entwicklung des Simulators wird ausschließlich
    Open-Source-Software eingesetzt. Dies fördert die Nachhaltigkeit
    durch Reduzierung von Lizenzkosten und Abhängigkeiten von
    proprietärer Software und unterstützt gleichzeitig die
    Open-Source-Gemeinschaft. Die Nutzung frei verfügbarer Ressourcen
    ermöglicht es, den Quellcode an spezifische Anforderungen
    anzupassen, was zu effizienteren und ressourcenschonenderen
    Anwendungen führt.

  \item \textbf{Nutzung vorhandener Hardware:} Es werden Raspberry Pi
    Einplatinencomputer verwendet, die bereits verfügbar sind. Durch
    die Wiederverwendung bestehender Hardware wird der Bedarf an neu
    produzierten elektronischen Geräten minimiert, was zur
    Reduzierung von Elektroschrott und zur Schonung natürlicher
    Ressourcen beiträgt. Diese Entscheidung ist sowohl ökologisch
    sinnvoll als auch kosteneffizient und unterstützt die
    Nachhaltigkeitsziele des Projekts.
\end{itemize}

\end{document}
