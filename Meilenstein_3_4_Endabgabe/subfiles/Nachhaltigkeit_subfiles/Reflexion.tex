\documentclass[../../main.tex]{subfiles} % Subfile-Class

\begin{document}

\subsection{Reflexion}

Während zu Beginn der Projektarbeit etwaige Nachhaltigkeitsaspekte
vernachlässigt
wurden, rückte der Nachhaltigkeitsgedanke gegen Ende der
Konzeptionsphase immer mehr in den Vordergrund.
Die anfängliche Vernachlässigung lässt sich hauptsächlich auf ein
Fehlen einer konkreten Lösung zurückführen.
Ohne handfeste Konzepte zur Realisierung des autonomen Fahrzeugs,
gestaltete es sich schwierig Punkte zu identifizieren,
wo wir nachhaltig handeln könnten. Erst als die Konzeptionen
konkreter wurden, konnten wir uns Massnahmen vorstellen,
die in die eventuelle Realisierung einfliessen könnten. Wir stellten
jedoch schnell fest, dass wir mit unserem limitierenden Budget nur
sehr geringfügig Entscheidungen treffen konnten, welche die
Nachhaltigkeit des Projekts signifikant beeinflussen würden.
Ein Punkt den wir hier hervorheben wollen ist, dass wir bei der
Beschaffung von Materialien und Komponenten
mehrheitlich nicht auf Übersee zurückgreifen. Diese Entscheidung
stand aber in starkem Konflikt mit dem
einzuhaltenden Budget. Ein weiterer Punkt warum sich die
Auseinandersetzung mit der Nachhaltigkeit auf
das Semesterende verschob, war die Tatsache, dass die operativen
Bewetungskritieren viel höher gewichtet werden.
Es wird schlussendlich nur einen minimalen Einfluss auf die Bewertung
haben, wenn wenn man sich für eine maximal nachhaltige Lösung entscheidet.
Die Abzüge bei Nichteinhalten der Anforderungen an das Fahrzeug sind
hier um einiges grösser.
Die investierte Zeit wird demnach priorisiert auf die Erfüllung der
Anforderungen gelegt, während die
Nachhaltigkeit als Nachgedanke behandelt wird.
Die Sensibilisierung für das Nachhaltigkeitsthema fanden wir dennoch
wertvoll. Eine seriöse Auseinandersetzung mit dem Thema
kann im richtigen Rahmen durchaus zu Lösungen führen, die welche
nicht nur ökologisch, sondern auch ökonomisch wertvoll sind.

\end{document}
