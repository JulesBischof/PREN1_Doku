\documentclass[main.tex]{subfiles} % Subfile-Class

% ============================================================================== %
%                            Subfile document                                    %
% ============================================================================== %

\begin{document}

% Template

\subsection{Hardfacts}

Die folgenden technischen Daten beschreiben die wichtigsten Eigenschaften und Leistungsparameter des entwickelten Fahrzeugs:

\begin{table}[h!]
    \centering
    \renewcommand{\arraystretch}{1.5}
    \begin{tabular}{|l|l|}
        \hline
        \textbf{Eigenschaft}            & \textbf{Wert} \\ \hline
        Maximale Geschwindigkeit        & 2.0 m/s       \\ \hline
        Energieversorgung               & 4S LiPo-Akku (14.4V, 2200 mAh) \\ \hline
        Betriebsdauer                   & ca. 20 Minuten (durchschnittlich) \\ \hline
        Maximale Leistung               & 30 W          \\ \hline
        Stromverbrauch im Leerlauf      & 5 W           \\ \hline
        Maximale Traglast der Greifeinheit & 200 g       \\ \hline
        Abmessungen                     & 30 cm x 30 cm x 50 cm \\ \hline
        Gewicht                         & 1.8 kg        \\ \hline
        Navigationsalgorithmus          & Heuristikbasierter DFS \\ \hline
        Sensoren                        & Liniensensor, Abstandssensor, Kamera \\ \hline
        Kommunikationsschnittstellen    & UART, I2C     \\ \hline
        Steuerungseinheit               & Raspberry Pi 5 \\ \hline
        Greifermechanik                 & Motorisierter Parallelgreifer \\ \hline
        Entwicklungstool für Simulation & Svelte.js-basierter Simulator \\ \hline
    \end{tabular}
    \caption{Technische Daten des autonomen Fahrzeugs}
    \label{tab:hardfacts}
\end{table}

\end{document}
