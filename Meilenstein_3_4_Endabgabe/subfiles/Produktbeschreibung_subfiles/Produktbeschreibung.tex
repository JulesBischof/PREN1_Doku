\documentclass[main.tex]{subfiles} % Subfile-Class

% ============================================================================== %
%                            Subfile document                                    %
% ============================================================================== %

\begin{document}

% Template

\subsection{Produktbeschreibung}

Das in diesem Projekt entwickelte System ist ein autonomes Fahrzeug, das in der Lage ist, sich selbstständig durch ein Wegenetzwerk zu navigieren. Es wurde speziell für die Aufgabe konzipiert, dynamisch auf Hindernisse und Einschränkungen zu reagieren, um den optimalen Weg von einem definierten Startpunkt zu einem Zielpunkt zu finden.

Das Fahrzeug zeichnet sich durch eine modulare Bauweise aus, die es ermöglicht, verschiedene Technologien effizient zu integrieren. Zu den zentralen Komponenten gehören:

\begin{itemize}
    \item \textbf{Chassis und Antrieb:} Das Fahrzeug verwendet ein leichtes, stabiles Chassis, das auf einem dreirädrigen Konzept basiert. Zwei Antriebsräder sorgen für die Fortbewegung, während ein stabilisierender Auflagepunkt zusätzliche Balance bietet. Die Antriebssteuerung erfolgt über Schrittmotoren, die eine präzise Bewegung und Manövrierfähigkeit gewährleisten.

    \item \textbf{Sensorik:} Ein Set aus Liniensensoren, Abstandssensoren und Kameras ermöglicht es dem Fahrzeug, die Leitlinien des Wegenetzwerks zu verfolgen, Hindernisse zu erkennen und die Position genau zu bestimmen. Diese Sensoren arbeiten zusammen, um eine Echtzeitbewertung der Umgebung durchzuführen.

    \item \textbf{Steuerungseinheit:} Ein Mikrocontroller-basiertes System übernimmt die Verarbeitung der Sensordaten und die Steuerung der Motoren. Durch die Integration eines speziell entwickelten Algorithmus wird das Fahrzeug in die Lage versetzt, dynamisch auf Veränderungen in der Umgebung zu reagieren.

    \item \textbf{Greifeinheit:} Für das Entfernen von Hindernissen ist das Fahrzeug mit einem motorisierten Greifer ausgestattet. Dieser kann Objekte sicher erfassen, anheben und präzise zurücklegen.

    \item \textbf{Energieversorgung:} Das System wird durch einen kompakten Lithium-Polymer-Akku betrieben, der eine ausreichende Betriebsdauer für die vorgegebene Aufgabenstellung sicherstellt. Zusätzliche Sicherheitsfunktionen wie eine Überwachung der Zellspannung und ein Not-Aus-Knopf garantieren die Betriebssicherheit.

    \item \textbf{Benutzerinterface:} Über einen Wahlschalter können vor dem Start die Zielpositionen ausgewählt werden. Visuelle und akustische Signale zeigen den Abschluss der Aufgabe an.

    \item \textbf{Simulator:} Zur Unterstützung der Entwicklung und zur frühzeitigen Validierung von Algorithmen hat das Informatik-Team einen Simulator in \textit{Svelte.js} entwickelt. Der Simulator ermöglicht es, das Verhalten des Fahrzeugs in einer virtuellen Umgebung zu testen, bevor physische Prototypen gebaut werden. Mit Funktionen wie der Simulation von Hindernissen, gesperrten Wegpunkten und alternativen Routen erlaubt der Simulator eine präzise Analyse der Navigations- und Steuerungsalgorithmen. Zudem bietet das Tool eine interaktive Benutzeroberfläche, um Szenarien flexibel zu konfigurieren und Ergebnisse in Echtzeit zu visualisieren.
\end{itemize}

Das Fahrzeug wurde so konzipiert, dass es sowohl den technischen Anforderungen als auch den strengen Gewichtsvorgaben gerecht wird. Dank seiner modularen Struktur und der klaren Trennung von Hardware- und Softwarekomponenten ist es flexibel erweiterbar und anpassungsfähig. Dies ermöglicht es, das System im weiteren Projektverlauf iterativ zu verbessern und zu optimieren.

\end{document}
