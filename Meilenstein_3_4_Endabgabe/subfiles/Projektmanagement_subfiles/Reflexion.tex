\documentclass[main.tex]{subfiles} % Subfile-Class


% ============================================================================== %
%                            Subfile document                                    %
% ============================================================================== %

\begin{document}

\subsection{Reflexion}
In einer gemeinsamen Diskussionsrunde wurde die Arbeitsweise bei der
Projektorganisation und -planung offen diskutiert und entsprechende Massnahmen
für das Folgemodul abgeleitet. Im Folgenden wird stichpunktartig festgehalten,
welche Punkte als positiv und welche als negativ bewertet wurden. Anschliessend
werden entsprechende Massnahmen für das Folgemodul festgehalten.

\subsubsection*{Was ist gut gelaufen}

\begin{itemize}
    \item Das Team ist mit dem Stand der Arbeit am Ende von PREN1 zufrieden. Ein
          realistisches Konzept wurde produktiv entwickelt.
    \item Es wurde viel bereits während des Semesters erarbeitet und die Dokumentation
          ist gegen Ende des Jahres bereits auf einem guten Stand. Es besteht daher kein
          Zeitdruck am Ende des Semesters, die Abschlussdokumentation fertig zu stellen.
          Ausserdem gibt es keine offenen Fragen mehr zur Lösung bestimmter Teilprobleme
          der Aufgabenstellung.
    \item Die Teamdynamik hat gut gepasst, es gab keine Streitigkeiten und es konnte
          produktiv an der Lösung der Aufgabe gearbeitet werden.
\end{itemize}

\subsubsection*{Was nicht so gut gelaufen}

\begin{itemize}
    \item Der Gesamtüberblick über den aktuellen Stand des Konzepts war nicht allen
          Teammitgliedern jederzeit klar.
    \item Die Einteilung der Planung in Sprints war bei der gewählten Arbeitsweise etwas
          überflüssig, da keine konkreten Ziele für das jeweilige Sprintende definiert
          wurden.
    \item Es wurde viel Zeit in die Entwicklung eines Greifers mit nur einem Motor
          investiert, der im Nachhinein nicht umsetzbar war. Eine einfachere Lösung war
          von Anfang an vorhanden und wird nun für das Projekt verwendet.
    \item Die Lernkurve einiger Teammitglieder für den Einstieg in die Dokumentation mit
          \LaTeX und die Zusammenarbeit über das Tool \textit{git} war etwas steil.
    \item Da der Zeitaufwand und die Arbeitspakete nicht von Anfang an klar waren, hatte
          die Vorplanung des Projekts den Charakter eines "Blindflugs", der durch die
          Meilensteine geführt wurde.

\end{itemize}

\subsection*{Fazit der Zusammenarbeit und Massnahmen für PREN2}
Da der Aufwand und die zu lösenden Teilaufgaben nicht ganz klar waren,
gestaltete sich die Planung etwas schwierig. Glücklicherweise gab es in der
Konzeptphase nur wenige Abhängigkeiten zwischen den einzelnen Fachbereichen und
Teilfunktionen, so dass jedes Teammitglied an seiner \textit{eigenen}
Teilfunktion weiterarbeiten und entsprechende \textit{ToDo's} vorplanen konnte.
Es ist zu erwarten, dass diese Art der Planung für das Nachfolgemodul PREN2
nicht zielführend und effizient sein wird, da beim Aufbau, der Inbetriebnahme
und dem Testen des Prototyps sehr viele Abhängigkeiten zwischen den einzelnen
Teilfunktionen bestehen.

Es ist nun klarer, welche Arbeitspakete bis zum fertigen Produkt existieren und
welche Abhängigkeiten auf dem Weg dorthin auftreten. Deshalb wird in den ersten
Wochen des neuen Semesters ein in Sprints aufgeteilter Wochenplan erstellt, in
dem klare Sprintziele definiert werden. Diese werden z.B. sein
\textit{Einzelteile für Prototyp beschafft}, \textit{Prototyp fahrfertig
    aufgebaut} oder etwa \textit{Firmware Antriebsregelung
    fertiggestellt}. Damit sind die Aktivitäten bis zum Sommer gut planbar und es
kann produktiv gearbeitet werden. Weiter wird dadurch vermieden, dass das
gesamte Projektteam aufgrund unterschiedlicher Vorlauf-/Wartezeiten auf
bestimmte Komponenten warten muss.

Der Umgang mit den Dokumentationswerkzeugen wie \LaTeX und \textit{git} hat
zwar anfangs zu Schwierigkeiten geführt - da der Umgang damit aber nun klar
ist, kann und soll dies auch so beibehalten werden.

\end{document}
