\documentclass[main.tex]{subfiles} % Subfile-Class
\usepackage{colortbl}
\usepackage[table]{xcolor} % Für erweiterte Farbauswahl
\definecolor{lightgray}{gray}{0.9}





% ============================================================================== %
%                            Subfile document                                    %
% ============================================================================== %

\begin{document}

% Template
In diesem Abschnitt werden das Finanz- und das Gewichtsbudget erläutert. Es wird auf die Finanzplanung
des Gesamtprojektes eingegangen sowie auf die getätigten Ausgaben in Produktentwicklung 1. Ausserdem wird 
das Gewicht des Roboters in allen Einzelteilen aufgelistet.


\subsection{Finanzplanung und Ausgaben}

Gemäss der Aufgabenstellung ist das zur Verfügung stehende finanzielle Budget 
für das gesamte Projekt
auf 500 CHF beschränkt. Das Modul Produktentwicklung 1 darf dabei maximal 200 CHF
beanspruchen. Die restlichen 300 CHF können im Folgemodul beantragt werden. 
Die Komponenten, welche aus dem privaten Umfeld stammen, 
müssen gemäss der Vorgabe zum halben Preis verrechnet werden. Aufgrund der limitierten 
finanziellen Mitteln ist eine sorgfältige Finanzplanung für das Gesamtprojekt bereits in
der Planungsphase unerlässlich. Im folgenden Kapitel 
wird auf die Finanzplanung für das Gesamtprojekt sowie auf bereits getätigte Ausgaben in
Produktentwicklung 1 eingegangen.

\subsubsection*{Finanzplanung}
Die Tabelle~\ref{tab:Budgetplanung} veranschaulicht die finanzielle Budgetplanung für das 
gesamte Projekt. Die Tabelle ist in Kategorien wie Antrieb, Greifeinheit, 
Chassis und Energieversorgung unterteilt, um eine bessere Übersicht zu gewährleisten. 
Die geschätzten Kosten basieren auf aktuellen Marktpreisen und enthalten auch 
Kleinteile sowie Leiterplatten. Gewisse Komponenten sind schon im Besitz von den
Entwicklern. Gemäss der Vorgabe werden diese Komponenten direkt mit dem halben
Marktpreis in der Tabelle angegeben. Eigenkonstruktionen sind mit einem realistischen 
Schätzwert angegeben. Zu beachten ist, dass sich die Preise ständig ändern. Die 
Tabelle dient daher als Richtwert.\\


\newpage

\begin{table}[h]
    \centering
    \scriptsize % Schriftgrösse verkleinern
    \begin{tabular}{|p{3cm}|p{2.5cm}|p{3cm}|p{1cm}|p{1cm}|p{1.5cm}|p{1cm}|}
        \hline
        \textbf{Komponente}                    & \textbf{Hersteller}   & \textbf{Hersteller Nr.} & \textbf{Besitz} & \textbf{Anzahl}& \textbf{Kosten [CHF/stk]} & \textbf{Kosten total [CHF]} \\ \hline
        \rowcolor{lightgray} Antrieb           &                       &                         &                 &                &                           &                             \\ \hline
        Schrittmotoren                         & DFRobot               & FIT0278                 & Nein            & 2              & 12.65                     & 25.3                          \\ \hline
        Schrittmotorentreiber                  & ADI-Trinamic          & TMC5240 Eval Board      & Ja              & 2              & 31.5                      & 63                          \\ \hline
        \rowcolor{lightgray} Greifeinheit      &                       &                         &                 &                &                           &                             \\ \hline
        Lichtschranke Hinderniss               & SICK                  & WL100-P3430             & Ja              & 1              & 15.25                     & 15.25                       \\ \hline
        Servo                                  & DFRobot               & SER0063                 & Nein            & 2              & 13.67                     & 27.34                       \\ \hline
        PCB                                    & PCBWay                & -                       & Nein            & 5              & 4                         & 4                           \\ \hline        
        \rowcolor{lightgray} Chassis           &                       &                         &                 &                &                           &                             \\ \hline
        LIDAR                                  & Benewake              & TF Luna I2C             & Nein            & 1              & 22.6                      & 22.6                        \\ \hline
        Liniensensor                           & Eigenkonstruktion     & -                       & Nein            & 1              & 40                        & 40                          \\ \hline
        Ultraschallsensor                      & OSEPP Electronics     & HC-SR04                 & Ja              & 1              & 1.55                      & 1.55                        \\ \hline
        PCB                                    & PCBWay                & -                       & Nein            & 5              & 7                         & 7                           \\ \hline
        Antriebsräder                          & DFRobot               & FIT0500                 & Nein            & 2              & 1.37                      & 2.74                        \\ \hline
        Laufkugeln                             & Polulu                & Polulu Item: 954        & Nein            & 1              & 4.5                       & 4.5                         \\ \hline
        Lichtschranken Encoder                 & Panasonic             & PM-F45-P                & Ja              & 2              & 5.75                      & 11.5                        \\ \hline
        \rowcolor{lightgray} Controller        &                       &                         &                 &                &                           &                             \\ \hline
        Raspberry Pi 4 2GB RAM                 & Raspberry Pi          & SC0193(9)               & Ja              & 1              & 26                        & 26                          \\ \hline
        Raspberry Pi Kamera 12MP               & Raspberry Pi          & RASPBERRY PI CAMERA 3   & Nein            & 1              & 25.35                     & 25.35                       \\ \hline
        Raspberry Pico                         & Raspberry Pi          & SC0915                  & Ja              & 2              & 1.8                       & 3.6                         \\ \hline
        PCB                                    & PCBWay                & -                       & Nein            & 5              & 7                         & 7                           \\ \hline
        \rowcolor{lightgray} Energieversorgung &                       &                         &                 &                &                           &                             \\ \hline
        Akkupack                               & Swaytronic            & 7640182625221           & Nein            & 1              & 41.7                      & 41.7                        \\ \hline
        Ladegerät                              & Voltcraft             & 1597950-UP              & Nein            & 1              & 39.95                     & 39.95                       \\ \hline
        PCB                                    & PCBWay                & -                       & Nein            & 1              & 7                         & 7                           \\ \hline
        \rowcolor{lightgray} Elektronik        &                       &                         &                 &                &                           &                             \\ \hline
        Kleinteile Elektronik                  & Elektro-Werkstatt HSLU& -                       & Nein            & -              & -                         & 40                          \\ \hline
        Sonstige Produktionskosten             & -                     & -                       & Nein            & -              & -                         & 20                          \\ \hline
        \textbf{Total}                         &                       &                         &                 &                &                           & \textbf{435.38}             \\ \hline
    \end{tabular}
    \caption{Finanzplanung}
    \label{tab:Budgetplanung}
\end{table}

Aus der vorliegenden Tabelle geht hervor, dass die Gesamtkosten 415,38 CHF betragen.
Dies liegt unter dem ursprünglich veranschlagten Budget von 500 CHF. 
Das resultierende Restbudget beträgt 64,62 CHF, welches für unvorhergesehene Ausgaben oder zusätzliche
Anschaffungen genutzt werden kann.

\subsubsection{Ausgaben Produktentwicklung 1}
Im Rahmen des Moduls Produktentwicklung 1 wurden die in
Tabelle~\ref{tab:Ausgaben_PREN1} aufgeführten Ausgaben getätigt. Es wird 
an dieser Stelle nochmals darauf hingewiesen, dass das zur Verfügung stehende 
Budget für Produktentwicklung 1 auf 200 CHF begrenzt ist. Die Komponenten, welche mit 
Eigenbesitz markiert sind, stammen direkt aus dem privaten Umfeld der Entwickler.
Bei diesen Komponenten ist bereits der halbe Marktpreis in der Tabelle eingetragen.


\newpage

\begin{table}[h]
    \centering
    \makebox[\textwidth][c]{ % Erzwingt horizontale Zentrierung
        \scriptsize % Schriftgrösse verkleinern
        \begin{tabular}{|p{3cm}|p{2.5cm}|p{2.5cm}|p{3cm}|p{1cm}|p{1.5cm}|p{1cm}|}
            \hline
            \textbf{Komponente}                     & \textbf{Lieferant} & \textbf{Hersteller}          & \textbf{Hersteller Nr.} & \textbf{Anzahl} & \textbf{Kosten [CHF/stk]} & \textbf{Kosten total [CHF]} \\ \hline
            \rowcolor{lightgray} Antrieb            &                    &                              &                         &                 &                           &                             \\ \hline
            Schrittmotoren                          & Digikey            & DFRobot                      & FIT0278                 & 2               & 12.65                     & 25.3                        \\ \hline
            %Schrittmotorentreiber                  & Eigenbesitz        & ADI-Trinamic                 & TMC5240 Eval Board      & 2               & 31.5                      & 63                          \\ \hline
            \rowcolor{lightgray} Greifeinheit       &                    &                              &                         &                 &                           &                             \\ \hline
            Lichtschranke                           & Eigenbesitz        & SICK                         & WL100-P3430             & 1               & 15.25                     & 15.25                       \\ \hline
            Servo                                   & Mouser             & DFRobot                      & SER0063                 & 2               & 13.67                     & 27.34                       \\ \hline
            \rowcolor{lightgray} Chassis            &                    &                              &                         &                 &                           &                             \\ \hline
            LIDAR                                   & Mouser             & Benewake                     & TF Luna I2C             & 1               & 22.6                      & 22.6                        \\ \hline
            Ultraschallsensor                       & Eigenbesitz        & OSEPP Electronics            & HC-SR04                 & 1               & 1.55                      & 1.55                           \\ \hline
            Antriebsräder                           & Mouser             & DFRobot                      & FIT0500                 & 2               & 1.37                      & 2.74                        \\ \hline
            %\rowcolor{lightgray} Controller        &                    &                              &                         &                 &                           &                                 \\ \hline
            %Raspberry Pi 4                         & ?                  &                              & ?                       & 1               & 17.5?                     & 17.5?                           \\ \hline
            %Raspberry Pi Kamera 12MP               & ?                  &                              & ?                       & 1               & 25.35                     & 25.35                           \\ \hline
            %Raspberry Pico                         & ?                  &                              & SC0915                  & 2               & 3.63                      & 3.53                            \\ \hline
            %PCB                                    & ?                  &                              &                         & 5               & 7                         & 7?                              \\ \hline
            %\rowcolor{lightgray} Energieversorgung &                    &                              &                         &                 &                           &                                 \\ \hline
            %Akkupack                               & Swaytronic         &                              & 7,64018E+12 ??          & 1               & 41.7                      & 41.7                            \\ \hline
            %Ladegerät                              & Voltcraft          &                              & 1597950-UP              & 1               & 39.95                     & 39.95                           \\ \hline
            %PCB                                    & ?                  &                              &                         & 1               & 7?                        & 7?                              \\ \hline
            \rowcolor{lightgray} Liniensensor       &                    &                              &                         &                 &                           &                             \\ \hline
            Trimmerpotentiometer 200Ohm SMD         & Mouser             & Vishay                       & TS3YJ201MR15            & 10              & 1.85                      & 18.5                        \\ \hline
            UV-Emitter 395nm                        & Digikey            & Bivar Inc.                   & UV3TZ-395-15            & 10              & 1.573                     & 15.73                       \\ \hline
            UV-Fototransistor 630nm                 & Digikey            & Everlight Electronics Co Ltd & ALS-PT204-6C/L177       & 10              & 0.496                     & 4.96                        \\ \hline
            IR-Emitter 940nm                        & Digikey            & würth Elektronik             & 15400394F3590           & 2               & 0.351                     & 0.702                       \\ \hline
            IR-Fototransistor 940nm                 & Digikey            & würth Elektronik             & 1540031NC6090           & 2               & 0.329                     & 0.658                       \\ \hline
            PCB-Liniensensor                        & PCB-Way            & PCBWay                       & -                       & 5               & 7                         & 7                           \\ \hline
            \rowcolor{lightgray} Produktion         &                    &                              &                         &                 &                           &                             \\ \hline
            3D-Druck PLA                            & Pauschale          & -                            & -                       & 0.2 Kg          & 23 CHF/Kg                 & 4.6                         \\ \hline
            3D-Druck PETG                           & Pauschale          & -                            & -                       & 0.2 Kg          & 23 CHF/Kg                 & 4.6                         \\ \hline

            \textbf{Total}                          &                    &                              &                         &                 &                           & \textbf{151.53}                \\ \hline
        \end{tabular}
    }
    \caption{Ausgaben PREN1}
    \label{tab:Ausgaben_PREN1}
\end{table}

Die Gesamtausgaben aus der Tabelle~\ref{tab:Ausgaben_PREN1} belaufen sich auf 151,53 CHF und liegen
damit innerhalb des vorgesehenen Budgets von 200 CHF. Mit diesen Komponenten 
konnten erste Prototypen entwickelt und grundlegende Funktionen getestet werden. Die Entwicklung 
des Liniensensors fällt finanziell höher aus als erwartet. Dies ist darauf zurückzuführen, dass der
Preis der UV-Emitter und der Trimmerpotentiometer bei der Budgetkalkulation unterschätzt wurden.

\end{document}
