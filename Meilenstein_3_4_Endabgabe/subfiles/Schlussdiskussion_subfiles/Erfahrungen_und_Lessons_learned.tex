\documentclass[main.tex]{subfiles} % Subfile-Class

\begin{document}

\subsection{Erfahrungen \& Lessons learned}

\subsubsection{Mechanik}
Text here

\subsubsection{Elektronik}
Text here

\subsubsection{Informatik}
Das Entwicklen von Software auf constrainter Hardware beansprucht die
Evaluation von
Faktoren, welche bei klassischen Programmieraufgaben nicht in
Betracht gezogen werden müssen.
Die Erfahrung und damit zu überwindenden Herausforderungen sind
wertvoll und ermöglichen eine differenzierte
Auseinandersitzung mit einem sonst kaum besprochenen Themengebiet in Rotkreuz.
Im Folgekurs PREN2 wird sich dieser Umstand sicherlich noch
massgebend auf die eventuelle Realisierung auswirken.
Die kollaborative Entwicklung des Simulators ging reibungslos von
statten, was auch dem Zustand geschuldet ist, dass wir
uns bereits früh gegenseitig auf unsere Stärken und Schwächen
geachtet und entsprechend geplant haben.
Aufgrund von diesen Planungen haben wir uns aber auch Mehraufwand in
PREN2 eingefahren. Der für den Simulator geschriebene
Code kann voraussichtlich nicht für das physische Fahrzeug
wiederverwendet werden. Daraus resultiert ein bemerkbarer
Aufwand beim Implementieren des Pfadfindungsalgorithmus, da der
geschriebene JavaScript Code für den Simulator manuell in
eine Sprache transpiliert werden muss, welche auf unseren Low -und
Highlevel Controllern effizient lauffähig ist.

\end{document}
