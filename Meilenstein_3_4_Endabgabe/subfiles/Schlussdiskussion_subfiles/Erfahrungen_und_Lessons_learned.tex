\documentclass[main.tex]{subfiles} % Subfile-Class

\begin{document}

\subsection{Erfahrungen \& Lessons learned}

\subsubsection{Mechanik}

Zu Beginn schien die Aufgabenstellung vergleichsweise unkompliziert und wenig zeitintensiv zu sein.
Im Laufe des Projekts zeigte sich jedoch, dass einige Aspekte komplexer waren als ursprünglich erwartet.
Die Ideenfindung und die Einigung auf ein Gesamtkonzept verliefen im Team insgesamt reibungslos.
Aus physischen Prototypen stellte sich heraus, dass ein einfacher Mechanismus mit zwei Motoren erheblich zuverlässiger
ist als ein komplexer Mechanismus, der nur einen Motor für die Greifereinheit verwendet.
Ein entscheidender Punkt, der während der Konstruktion unterschätzt wurde, war der Aufwand für die Entwicklung
des Chassis. Die Platzierung der verschiedenen Komponenten im begrenzten Bauraum in Abstimmung mit den 
Elektrotechnikern erwies sich als deutlich zeitaufwendiger als angenommen.
Besonders herausfordernd war es, das maximale Gewicht des Systems einzuhalten, ein Aspekt,
der im PREN2 ein zentraler Schwerpunkt sein wird


\subsubsection{Elektronik}
Die Entwicklung eines Systems, das für viele mögliche Lösungen offen sein soll,
ist schwierig, wenn nicht klar ist, welchen Umfang das System später haben
wird. Konkret geht es um die Anzahl und Form der benötigten
Sensorschnittstellen und Kommunikationsschnittstellen.Was zu Anfang stark
unterschätzt wurde, ist der Energiebedarf, den ein Computer wie der Raspberry
Pi 5 mit sich bringt. Der Raspberry Pi 5 mit einer Leistungsaufnahme von 25W
ist ein enormer Stromverbraucher in unserem System, was angesichts seiner
Grösse überrascht. Ausserdem sind die Preisunterschiede bei der Vergabe von
Fertigungsaufträgen an asiatische Leiterplattenhersteller erheblich. Werden die
exakten Baumaße von $100$ mm × $100$ mm sowie die vorgegebenen Via-Größen nicht
eingehalten, steigen die Preise für diese vermeintlich günstigen Leiterplatten rasant.

Die Entwicklung des Liniensensors hat viel Zeit in Anspruch genommen und zu
einigen spannenden Erkenntnissen geführt, als die Ströme für verschiedene
Lichtspektren verglichen wurden. Es war von entscheidender Bedeutung, sich
nicht auf einen Lösungsansatz festzusetzen, sondern lösungsoffen für mehrere
Optionen zu sein. Die Entscheidung, die digitale Auswertung der Entladezeit
eines Kondensators durch die analoge Auswertung eines Spannungspegels zu
ersetzen, hat sich als ausserordentlich vorteilhaft erwiesen, da so wesentlich
schnellere Messzeiten zu erreichen sind. Diskussionen mit den anderen Teams
erweisen sich als sehr spannend, da alle vor den gleichen Problemen stehen,
aber eigene Lösungsansätze verfolgen.

\subsubsection{Informatik}
Das Entwicklen von Software auf constrainter Hardware beansprucht die
Evaluation von Faktoren, welche bei klassischen Programmieraufgaben nicht in
Betracht gezogen werden müssen. Die Erfahrung und damit zu überwindenden
Herausforderungen sind wertvoll und ermöglichen eine differenzierte
Auseinandersitzung mit einem sonst kaum besprochenen Themengebiet in Rotkreuz.
Im Folgekurs PREN2 wird sich dieser Umstand sicherlich noch massgebend auf die
eventuelle Realisierung auswirken. Die kollaborative Entwicklung des Simulators
ging reibungslos von statten, was auch dem Zustand geschuldet ist, dass wir uns
bereits früh gegenseitig auf unsere Stärken und Schwächen geachtet und
entsprechend geplant haben. Aufgrund von diesen Planungen haben wir uns aber
auch Mehraufwand in PREN2 eingefahren. Der für den Simulator geschriebene Code
kann voraussichtlich nicht für das physische Fahrzeug wiederverwendet werden.
Daraus resultiert ein bemerkbarer Aufwand beim Implementieren des
Pfadfindungsalgorithmus, da der geschriebene JavaScript Code für den Simulator
manuell in eine Sprache transpiliert werden muss, welche auf unseren Low -und
Highlevel Controllern effizient lauffähig ist.

\end{document}
