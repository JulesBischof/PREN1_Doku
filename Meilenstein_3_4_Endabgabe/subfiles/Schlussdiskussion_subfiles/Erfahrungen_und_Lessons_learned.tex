\documentclass[main.tex]{subfiles} % Subfile-Class

\begin{document}

\subsection{Erfahrungen \& Lessons learned}

\subsubsection{Mechanik}
Text here

\subsubsection{Elektronik}
Die Entwicklung eines Systems, das für viele mögliche Lösungen offen sein soll,
erweist sich als schwierig, wenn nicht klar ist, welchen Umfang das System
später haben soll. Damit ist gemeint, in welcher Anzahl und Form
Sensorschnittstellen und Kommunikationsschnittstellen benötigt werden. Was
anfangs stark unterschätzt wurde, ist der Energiebedarf, den ein Computer wie
der Raspberry Pi mit sich bringt. So ist der Raspberry Pi 5 mit einer
Leistungsaufnahme von 25W ein enormer Stromverbraucher in unserem System, was
angesichts der Grösse dieses Computers überrascht hat. Des Weiteren sind die
grossen Preisunterschiede bei der Vergabe von Fertigungsaufträgen auch an
asiatische Leiterplattenhersteller gross. Werden die exakten Baumasse von $100
    \cdot 100$ mm sowie die vorgegebenen Via-Grössen nicht eingehalten, werden die
Preise für diese Leiterplatten sehr plötzlich ebenfalls teurer.

Die Entwicklung des Liniensensors hat viel Zeit in Anspruch genommen und zu
einigen spannenden Aha-Effekten geführt, als die Ströme für verschiedene
Lichtspektren verglichen wurden. Es hat auch geholfen, sich nicht strikt auf
einen Lösungsansatz zu versteifen, sondern lösungsoffen für mehrere Optionen zu
sein. So hat sich die Umentscheidung von einer digitalen Auswertung der
Entladezeit eines Kondensators zur Analogen Auswertung eines Spannungspegels
als sehr vorteilhaft erwiesen, da so wesentlich schnellere Messzeiten zu
erreichen sind. Gleichzeitig war die Abstimmung und Diskussion mit den anderen
Teams spannend, da alle gemeinsam vor dem gleichen Problem standen und eigene
Lösungsansätze für dieses verfolgt haben.

\subsubsection{Informatik}
Das Entwicklen von Software auf constrainter Hardware beansprucht die
Evaluation von Faktoren, welche bei klassischen Programmieraufgaben nicht in
Betracht gezogen werden müssen. Die Erfahrung und damit zu überwindenden
Herausforderungen sind wertvoll und ermöglichen eine differenzierte
Auseinandersitzung mit einem sonst kaum besprochenen Themengebiet in Rotkreuz.
Im Folgekurs PREN2 wird sich dieser Umstand sicherlich noch massgebend auf die
eventuelle Realisierung auswirken. Die kollaborative Entwicklung des Simulators
ging reibungslos von statten, was auch dem Zustand geschuldet ist, dass wir uns
bereits früh gegenseitig auf unsere Stärken und Schwächen geachtet und
entsprechend geplant haben. Aufgrund von diesen Planungen haben wir uns aber
auch Mehraufwand in PREN2 eingefahren. Der für den Simulator geschriebene Code
kann voraussichtlich nicht für das physische Fahrzeug wiederverwendet werden.
Daraus resultiert ein bemerkbarer Aufwand beim Implementieren des
Pfadfindungsalgorithmus, da der geschriebene JavaScript Code für den Simulator
manuell in eine Sprache transpiliert werden muss, welche auf unseren Low -und
Highlevel Controllern effizient lauffähig ist.

\end{document}
