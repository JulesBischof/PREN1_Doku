\documentclass[main.tex]{subfiles} % Subfile-Class


% ============================================================================== %
%                            Subfile document                                    %
% ============================================================================== %

\begin{document}

% Template

\subsection{Fazit}

Das PREN 1-Projekt ermöglichte es unserem Team, die Grundlagen für die  
Entwicklung eines autonomen Roboters zu schaffen. In der frühen Phase der  
Ideenentwicklung war es naturgemäss herausfordernd, fachübergreifende  
Abhängigkeiten im Voraus vollständig zu planen. Viele Verknüpfungen zwischen  
Mechanik, Elektronik und Informatik wurden erst im Projektverlauf sichtbar,  
wodurch die Arbeitsweise häufig reaktiv gestaltet werden musste.\\

Trotz dieser Herausforderungen konnten wir ein solides Gesamtkonzept entwickeln,  
das den Anforderungen gerecht wird und die Grundlage für die nächste  
Projektphase legt. Die Fähigkeit, flexibel auf neue Situationen zu reagieren,  
sowie die kreative Problemlösung im Team waren entscheidend für den Erfolg  
in diesem Modul. Gleichzeitig wurde deutlich, dass für PREN 2 eine gezielte  
Verbesserung in der Planung und Abstimmung erforderlich ist, um effizienter  
zu arbeiten und Synergien besser zu nutzen. 

\subsubsection{Erfüllung der Anforderungen}  
Der Erfüllungsgrad der Anforderungen in PREN 1 zeigt, dass viele zentrale Ziele  
erreicht wurden. Dennoch gibt es einige offene Punkte, insbesondere bei den  
Festanforderungen, die in PREN 2 priorisiert bearbeitet werden müssen 
(siehe Anhang~\ref{appendix:Erfüllung_Anforderungsliste}).

Zu den offenen Festanforderungen gehören vor allem das Gewicht und die Sicherheit.  
Das System liegt aktuell knapp innerhalb des Gewichtslimits, jedoch ist eine gezielte  
Optimierung notwendig, um Spielraum für zukünftige Anpassungen zu schaffen.  
Zusätzlich fehlen wichtige Sicherheitsmassnahmen, wie robuste Notfallmechanismen,  
die für einen sicheren Betrieb unverzichtbar sind.

Eine für die Testphase besonders relevante Wunschanforderung ist die Akkulaufzeit.  
Diese konnte in PREN 1 noch nicht ausreichend evaluiert werden, was im nächsten  
Modul priorisiert erfolgen sollte, um realistische Tests des Systems zu ermöglichen.  
Weitere Wunschanforderungen, wie die Zeitauswertung während des Parcours oder  
die Debug-Schnittstelle, bleiben sekundär und können je nach Projektverlauf  
ergänzt oder verworfen werden.

Zusammenfassend zeigt der Erfüllungsgrad der Anforderungen, dass das Projekt auf  
einem soliden Fundament steht. Die offenen Punkte bieten klare Handlungsfelder  
für PREN 2 und können gezielt weiterentwickelt werden, um die vollständige  
Erfüllung aller Anforderungen sicherzustellen.

\subsubsection{Ausblick und Risiken}  
Basierend auf den Erfahrungen aus PREN 1 ergeben sich für PREN 2 konkrete  
Handlungsfelder sowie potenzielle Risiken, die es zu adressieren gilt:  

\subsubsection*{Elektronik}  
\begin{itemize}  
    \item \textbf{Leiterplatten bestellen und zusammenbauen:}  
    Verzögerungen bei der Beschaffung könnten die Inbetriebnahme der  
    Elektronik verzögern. Eine rechtzeitige Bestellung ist daher essenziell.  
    \item \textbf{Firmwareentwicklung:}  
    Die Firmware für die Steuerung der Boards und Antriebe muss entwickelt  
    und getestet werden, wobei Fehler in der Programmierung die Funktionalität  
    beeinträchtigen könnten.  
    \item \textbf{Strombedarf evaluieren:}  
    Eine fehlerhafte Bewertung des Strombedarfs könnte zu Problemen mit der  
    Energieeffizienz oder einer unzureichenden Batteriekapazität führen.  
\end{itemize}  

\subsubsection*{Mechanik}  
\begin{itemize}  
    \item \textbf{Überarbeitung von Chassis und Gewicht:}  
    Das Design des Chassis muss überarbeitet werden, um das  
    Gewicht zu optimieren.
    \item \textbf{Prototyp fertigen und testen:}  
    Ein früher Prototyp ist notwendig, um potenzielle Probleme rechtzeitig zu  
    identifizieren. Verzögerungen könnten die Tests und die Integration mit  
    anderen Disziplinen beeinträchtigen. 
    \item \textbf{Räder mit unzureichendem Grip:}  
    Sollten die Räder nicht genügend Traktion bieten, könnte dies die Navigation  
    und Manövrierfähigkeit des Roboters negativ beeinflussen.  
\end{itemize}  

\subsubsection*{Informatik}  
\begin{itemize}  
    \item \textbf{Karten erstellen und optimieren:}  
    Die Entwicklung einer präzisen Kartierungsstrategie  
    ist essenziell. Fehlerhafte Karten könnten die erfolgreiche Navigation behindern.  
    \item \textbf{Charakter- und Abgangserkennung:}  
    Die Kamerafunktionen zur Erkennung von Wegabgängen und Buchstaben müssen  
    umfangreich getestet werden. Unzureichende Ergebnisse könnten die  
    Zielerreichung gefährden.  
    \item \textbf{Algorithmusanpassung:}  
    Der Algorithmus muss auf eine ressourcenschonende Sprache für den Raspberry Pi  
    umgeschrieben werden, um Speicher- und Rechenressourcen effizient zu nutzen.
\end{itemize}  

\subsubsection*{Allgemeines}  
\begin{itemize}  
    \item \textbf{Projektplanung verbessern:}  
    Massnahmen zur besseren Abstimmung zwischen den Disziplinen sind entscheidend,  
    um Leerlaufzeiten zu minimieren. Eine unzureichende Planung könnte Verzögerungen  
    verursachen.  
    \item \textbf{Abhängigkeiten erkennen:}  
    Die Identifikation und Berücksichtigung fachübergreifender Abhängigkeiten ist  
    notwendig, damit alle Teilbereiche parallel arbeiten können.  
\end{itemize}  

Durch die frühzeitige Berücksichtigung dieser Risiken und die Umsetzung der genannten  
Massnahmen wird es möglich sein, die offenen Punkte effizient zu bearbeiten und den  
Roboter in PREN 2 erfolgreich weiterzuentwickeln.

\end{document}
