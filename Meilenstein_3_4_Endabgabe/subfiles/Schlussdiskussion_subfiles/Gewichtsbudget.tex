\documentclass[main.tex]{subfiles} % Subfile-Class


% ============================================================================== %
%                            Subfile document                                    %
% ============================================================================== %

\begin{document}

\subsection{Gewichtsbudget}

Gemäss der Aufgabenstellung ist das Gewicht des Fahrzeugs auf 2 kg begrenzt.
Daher ist eine sorgfältige Gewichtsanalyse bereits in der Planungsphase von entscheidender Bedeutung.
In Tabelle~\ref{tab:Gewichtsbudget} sind die Gewichte der verschiedenen Komponenten aufgeführt.
Für Zukaufteile wurden die Gewichte den Herstellerdaten entnommen, während bei Eigenkonstruktionen die 
Gewichtswerte anhand von Abschätzungen im CAD unter Berücksichtigung des entsprechenden Materials berechnet wurden.

\begin{table}[h!]
    \centering
    \scriptsize % Schriftgrösse verkleinern
    \begin{tabularx}{\textwidth}{|>{\raggedright\arraybackslash}p{3cm}|>{\raggedright\arraybackslash}p{3cm}|>{\raggedright\arraybackslash}p{2cm}|>{\centering\arraybackslash}p{1.5cm}|>{\centering\arraybackslash}p{1.5cm}|>{\centering\arraybackslash}p{1.5cm}|}
        \hline
        \textbf{Beschreibung}            & \textbf{Hersteller} & \textbf{Hersteller Nr.} & \textbf{Anzahl} & \textbf{Gewicht [g]/stk} & \textbf{Gewicht total[g]}\\ \hline
        Schrittmotoren                   & DFRobot             & FIT0278                 & 2               & 287                     & 574                      \\ \hline
        Schrittmotorentreiber            & DFRobot             & DRI0043                 & 2               & 38                      & 76                       \\ \hline
        Servomotor                       & ~                   & ~                       & 1               & 80                      & 80                       \\ \hline
        Akkupack                         & Swaytronic          & ~                       & 1               & 130                     & 130                      \\ \hline
        LCD                              & ~                   & ~                       & 0               & 70                      & 0                        \\ \hline
        zusätzliche PCBs                 & JLCPCB              & ~                       & 5               & 60                      & 300                      \\ \hline
        Sensorpauschale                  & diverse             & ~                       & 9               & 10                      & 90                       \\ \hline
        Microcontroller                  & Raspberry Pico      & ~                       & 2               & 4                       & 8                        \\ \hline
        Controller                       & Raspberry Pi 5      & ~                       & 1               & 46                      & 46                       \\ \hline
        Räder                            & Pauschale           & ~                       & 2               & 40                      & 80                       \\ \hline
        Chassis                          & Pauschale           & ~                       & 1               & 200                     & 200                      \\ \hline
        Greifeinheit                     & ~                   & ~                       & 1               & 100                     & 100                      \\ \hline
        Verkabelungspauschale            & ~                   & ~                       & 1               & 100                     & 100                      \\ \hline
        Schraubenpauschale               & ~                   & ~                       & 1               & 50                      & 50                       \\ \hline
        \textbf{Summe}                   &                     &                         &                 &                         & \textbf{1811}            \\ \hline
    \end{tabularx}
    \caption{Gewichtsbudget}~\label{tab:Gewichtsbudget}
\end{table}
\end{document}
