\documentclass[main.tex]{subfiles} % Subfile-Class


% ============================================================================== %
%                            Subfile document                                    %
% ============================================================================== %

\begin{document}

\subsection{Gewichtsbudget}

Gemäss der Aufgabenstellung ist das Gewicht des Fahrzeugs auf 2 Kilogramm
begrenzt. Daher ist eine sorgfältige Gewichtsanalyse bereits in der
Planungsphase von entscheidender Bedeutung. In Tabelle~\ref{tab:Gewichtsbudget}
sind die Gewichte der verschiedenen Komponenten aufgeführt. Für Zukaufteile
wurden die Gewichte den Herstellerdaten entnommen, während bei
Eigenkonstruktionen die Gewichtswerte anhand von Abschätzungen im CAD unter
Berücksichtigung des verwendeten Materials berechnet wurden. 
Die Gewichtsanalyse zeigt, dass das Gewicht mit 1.996 Kilogramm knapp bemessen wurde.

\begin{table}[H]
    \centering
    \scriptsize % Schriftgrösse verkleinern
    \begin{tabularx}{\textwidth}{|>{\raggedright\arraybackslash}p{3cm}|>{\raggedright\arraybackslash}p{3cm}|>{\raggedright\arraybackslash}p{2cm}|>{\centering\arraybackslash}p{1.5cm}|>{\centering\arraybackslash}p{1.5cm}|>{\centering\arraybackslash}p{1.5cm}|}
        \hline
        \textbf{Komponente}             & \textbf{Hersteller} & \textbf{Hersteller Nr.} & \textbf{Anzahl} & \textbf{Gewicht [g]/stk} & \textbf{Gewicht total[g]} \\ \hline
        \rowcolor{lightgray} Motor      &                     &                         &                 &                          &                           \\ \hline
        Schrittmotoren                  & DFRobot             & FIT0278                 & 2               & 287                      & 574                       \\ \hline
        Schrittmotorentreiber           & ADI-Trinamic        & TMC5240 Eval Board      & 2               & 38                       & 76                        \\ \hline
        Servomotor                      & DFRobot             & SER0063                 & 2               & 80                       & 160                       \\ \hline
        \rowcolor{lightgray} Elektronik &                     &                         &                 &                          &                           \\ \hline
        Akkupack                        & Swaytronic          & 7640182625221           & 1               & 167                      & 167                       \\ \hline
        LCD                             & -                   & -                       & 0               & 70                       & 0                         \\ \hline
        zusätzliche PCBs                & PCBWay              & -                       & 4               & 60                       & 240                       \\ \hline
        Sensorpauschale                 & diverse             & -                       & 9               & 10                       & 90                        \\ \hline
        Microcontroller                 & Raspberry Pico      & SC0915                  & 2               & 4                        & 8                         \\ \hline
        Controller                      & Raspberry Pi 4      & SC0193(9)               & 1               & 46                       & 46                        \\ \hline
        \rowcolor{lightgray} Mechanik   &                     &                         &                 &                          &                           \\ \hline
        Räder                           & Pauschale           & -                       & 2               & 40                       & 80                        \\ \hline
        Chassis                         & Eigenkonstruktion   & -                       & 1               & 200                      & 200                       \\ \hline
        Greifeinheit                    & Eigenkonstruktion   & -                       & 1               & 100                      & 175                       \\ \hline
        Verkabelungspauschale           & -                   & -                       & 1               & 100                      & 100                       \\ \hline
        Schraubenpauschale              & -                   & -                       & 1               & 50                       & 50                        \\ \hline
        \textbf{Summe}                  &                     &                         &                 &                          & \textbf{1996}             \\ \hline
    \end{tabularx}
    \caption{Gewichtsbudget}~\label{tab:Gewichtsbudget}
\end{table}

Während der Entwicklung wurde deutlich, dass die Greifereinheit und das Chassis
schwerer sind als ursprünglich im Gewichtsbudget angenommen. In der nächsten Iteration
wird das Chassis komplett im 3D-Druck gefertigt, vermutlich mit einem Wabenmuster, 
um Gewicht zu sparen. Die PCBs werden in einem kompakten Turm mithilfe von
Disanzbuchsen übereinandergestapelt, wodurch mehrere Stützen und 3D-Druck
Platten eingespart werden. Jedoch werden diese Gewichtsoptimierungen erst im
PREN2 umgesetzt.

\end{document}
