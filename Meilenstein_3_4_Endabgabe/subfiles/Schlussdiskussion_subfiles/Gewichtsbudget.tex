\documentclass[main.tex]{subfiles} % Subfile-Class


% ============================================================================== %
%                            Subfile document                                    %
% ============================================================================== %

\begin{document}

\subsection{Gewichtsbudget}

Gemäss der Aufgabenstellung ist das Gewicht des Fahrzeugs auf 2 Kilogramm
begrenzt. Daher ist eine sorgfältige Gewichtsanalyse bereits in der
Planungsphase von entscheidender Bedeutung. In Tabelle~\ref{tab:Gewichtsbudget}
sind die Gewichte der verschiedenen Komponenten aufgeführt. Für Zukaufteile
wurden die Gewichte den Herstellerdaten entnommen, während bei
Eigenkonstruktionen die Gewichtswerte anhand von Abschätzungen im CAD unter
Berücksichtigung des verwendeten Materials berechnet wurden. Die
Gewichtsanalyse zeigt, dass das Fahrzeug knapp unter der Gewichtslimite ist,
leider ist diese zu knapp bemessen.

\begin{table}[H]
    \centering
    \scriptsize % Schriftgrösse verkleinern
    \begin{tabularx}{\textwidth}{|>{\raggedright\arraybackslash}p{3cm}|>{\raggedright\arraybackslash}p{3cm}|>{\raggedright\arraybackslash}p{2cm}|>{\centering\arraybackslash}p{1.5cm}|>{\centering\arraybackslash}p{1.5cm}|>{\centering\arraybackslash}p{1.5cm}|}
        \hline
        \textbf{Komponente}             & \textbf{Hersteller} & \textbf{Hersteller Nr.} & \textbf{Anzahl} & \textbf{Gewicht [g]/stk} & \textbf{Gewicht total[g]} \\ \hline
        \rowcolor{lightgray} Motor      &                     &                         &                 &                          &                           \\ \hline
        Schrittmotoren                  & DFRobot             & FIT0278                 & 2               & 287                      & 574                       \\ \hline
        Schrittmotorentreiber           & DFRobot             & DRI0043                 & 2               & 38                       & 76                        \\ \hline
        Servomotor                      & ~                   & ~                       & 2               & 80                       & 160                       \\ \hline
        \rowcolor{lightgray} Elektronik &                     &                         &                 &                          &                           \\ \hline
        Akkupack                        & Swaytronic          & ~                       & 1               & 167                      & 167                       \\ \hline
        LCD                             & ~                   & ~                       & 0               & 70                       & 0                         \\ \hline
        zusätzliche PCBs                & JLCPCB              & ~                       & 4               & 60                       & 240                       \\ \hline
        Sensorpauschale                 & diverse             & ~                       & 9               & 10                       & 90                        \\ \hline
        Microcontroller                 & Raspberry Pico      & ~                       & 2               & 4                        & 8                         \\ \hline
        Controller                      & Raspberry Pi 5      & ~                       & 1               & 46                       & 46                        \\ \hline
        \rowcolor{lightgray} Mechanik   &                     &                         &                 &                          &                           \\ \hline
        Räder                           & Pauschale           & ~                       & 2               & 40                       & 80                        \\ \hline
        Chassis                         & Pauschale           & ~                       & 1               & 200                      & 200                       \\ \hline
        Greifeinheit                    & ~                   & ~                       & 1               & 100                      & 175                       \\ \hline
        Verkabelungspauschale           & ~                   & ~                       & 1               & 100                      & 100                       \\ \hline
        Schraubenpauschale              & ~                   & ~                       & 1               & 50                       & 50                        \\ \hline
        \textbf{Summe}                  &                     &                         &                 &                          & \textbf{1996}             \\ \hline
    \end{tabularx}
    \caption{Gewichtsbudget}~\label{tab:Gewichtsbudget}
\end{table}

Während der Entwicklung wurde deutlich, dass die Greifereinheit und das Chassis
schwerer sind als ursprünglich im Gewichtsbudget angenommen. In der nächsten
Iteration wird das Chassis vollständig im 3D-Druckverfahren gefertigt,
voraussichtlich mit einem Wabenmuster, um durch gezielte Aussparungen Gewicht
zu reduzieren. Die PCBs werden in einem kompakten Turm mithilfe von
Disanzbuchsen übereinandergestapelt, wodurch mehrere Stützen und 3D-Druck
Platten eingespart werden. Jedoch werden diese Gewichtsoptimierungen erst im
PREN2 umgesetzt.

\end{document}
