% !TEX root = ../main.tex
\documentclass[../main.tex]{subfiles}

\setlength{\extrarowheight}{1pt}    % Adds extra padding to rows
\setlength{\arrayrulewidth}{0.2pt}  % Adjusts the thickness of the hline

\begin{document}

\noindent
Die nachfolgende Quellensammlung in \autoref{tab:sources} dient als Übersicht
zur Technologierecherche und wird im Laufe des Projekts
weitergeführt, um die verwendeten Quellen im Ausblick auf die
Schlussdokumentation zu sammeln. Die unter Grade aufgeführten Werte
dienen zur Bewertung der Relevanz der Quellen für das Projekt und
deren weiterführende Benutzung.

\subsection{Quellen}

\begin{table}[htbp]
  \centering
  \begin{tabularx}{\textwidth}{
      L{2cm}
      L{2.5cm}
      C{1cm}
      L{2.5cm}
      X
    }
    \toprule
    \textbf{Thema} & \textbf{Stichwort} & \textbf{Grade} &
    \textbf{Quelle} & \textbf{Beschreibung} \\
    \midrule
    \addlinespace[3pt]
    Mathematik & Algebra & 9 & Buch A & Eine Einführung in die
    Algebra mit ausführlichen Beispielen. \\
    \addlinespace[3pt]
    \hline
    \addlinespace[3pt]
    Physik & Mechanik & 8 & Buch B & Grundlegende Konzepte der
    Mechanik werden erläutert. \\
    \addlinespace[3pt]
    \hline
    \addlinespace[3pt]
    Informatik & Programmierung & 7 & Online-Kurs D & Grundlagen
    der Programmierung in Python für Einsteiger. \\
    \addlinespace[3pt]
    \bottomrule
  \end{tabularx}
  \caption{Quellensammlung}
  \label{tab:sources}
\end{table}

\end{document}
